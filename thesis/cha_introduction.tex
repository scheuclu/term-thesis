\chapter{Introduction}\label{cha:introduction}
\section{Motivation}\label{sec:motivation}
The field of high performance computing has seen a significant paradigm shift over the past 10-15 years. While Moore's law \cite{Moore1965} is still valid to this day, the way progress is achieved has drastically changed. Single-core performance has been stagnating for years, which gave rise to the advent of multi-core processors. But as the architectures are changing, so do the algorithms. Amdahl's law\cite{Rodgers1985} poses harsh requirements on efficient parallel algorithms. In order to reach satisfying scalability, inter-processor communication, as well as serial algorithm parts, have to be kept at an absolute minimum.
An intuitive way of approaching that goal is domain decomposition(DD). In particular, the Finite Element Tearing and Interconnection (FETI) or the Balanced Domain Decomposition (BDD) are well-established methods for mechanical problems.
The basic idea of DD methods is to split the domain into, ideally, equal-sized subdomains(substructures). Local Problems can then be solved for each substructure simultaneously on multiple cores sequentially, before an iterative technique is required to connect the domains together by finding the common interface unknowns.\\
Generally, primal and dual DD algorithms can be differentiated by the type of interface quantity that are used for the connection. Primal DD methods use interface displacements, whereas dual DD methods solve for the interface forces. A profound overview over these methods can be found in \cite{Gosselet2006}.\\
Unfortunately, classical DD shows very bad behaviour for cases with irregular-shaped substructures as well as cases with jumps in the material parameters along, as well as across the interface. For a general purpose algorithm, useful for real engineering problems, this has to be addressed. Several approaches have been proposed, based on scaling of the operators \cite{DaVeiga2014} or by solving for critical eigenmodes on the interface \cite{Spillane2013, Spillane2014}.\\
For FETI-type solvers, a promising  algorithm, the Simultaneous FETI (FETI-S), has been proposed in \cite{RixenPhD} for two sub-domains and was generalized for an arbitrary number of sub-domains in \cite{Rixen2013}.
An efficient implementation of this algorithm, as well as the Block FETI method (FETI-B) has recently been described \cite{Gosselet2015}.\\
A recent publication \cite{Spillane2016} has proposed a new, Adaptive Multi Preconditioned Conjugate Gradient Algorithm (AMPCG) for cases where the preconditioner is built as a sum of contributions (like in FETI). A first application to the Balancing Domain Decomposition method has shown promising results.
\section{Objective}\label{sec:obective}
This thesis aims at a profound analysis of the before mentioned, existing FETI algorithms. Their derivations will be described and theoretical aspects are going to be discussed. Intensive numerical assessments will highlight the potentials and weak points of each method and will be used as basis for the final conclusions.\\
Additionally, the before mentioned AMPCG algorithm will be applied on the FETI method and compared to the existing algorithms.\\
Finally, an outlook regarding further developments of the FETI methods will be drawn, and promising ideas will be discussed.
