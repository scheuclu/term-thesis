\documentclass[12pt]{AMdocumentation}
\usepackage{AMfont}
\usepackage{filecontents}

\providecommand{\AMlatex}{AM-\LaTeX{}\xspace}
\renewcommand{\maketitle}{\OLDmaketitle}

\author{Romain Pennec}
\title{How to Install \AMlatex}

\begin{filecontents}{biblio.bib}
@online{HOWTOdtx,
title = {How to compile the dtx files},
date = {2015},
url = {http://info/info/pmwiki_files/How-To-compile-dtx-files.pdf},
}
@online{HOWTOtexlive,
title = {How to upgrade texlive},
date = {2015},
url = {http://info/info/pmwiki_files/How-To-upgrade-texlive.pdf},
}
@online{HOWTOinstall,
title = {How to install AMlatex},
date = {2015},
url = {http://info/info/pmwiki_files/How-To-Install-AMlatex.pdf},
}
\end{filecontents}

\addbibresource{biblio.bib}
\nocite{HOWTOinstall}

\begin{document}
\maketitle

\vspace{1cm}
\tableofcontents

\section{From a release}
\label{sec:release}

The last release can be found in \url{transfer/03_latex/releases/} or on the department intranet (\url{info}).  After extraction of the archive, you should find a directory containing this pdf, the script  \FileName{setup.sh} (for Linux) and a \FileName{texmf} tree. Note also that a \FileName{texmf} tree corresponding to the last release is directly available at the address \url{R:/AMlatex/texmf}. Regardless of the chosen release, we must make the system aware of the \FileName{texmf} tree.

\subsection{Installation for Linux}
\label{sec:release:linux}

Assuming you have a recent \texlive distribution, you have two possibilities: either you copy 
\FileName{texmf} in a standard location (like \verb|$HOME| or \verb|/usr/local/share|) and run 
\cmdbox*{texhash}, or you add the path of the \AMlatex texmf tree to the variable \EnvVariable{TEXMFLOCAL}.
Both possibilities are implemented in the script \FileName{setup.sh}, so what you have to do is:

\begin{itemize}
\item Open a terminal and go in the directory \FileName{AMlatex} obtained from the extraction of the release archive.
\item Enter \cmdbox{./setup.sh install}.
\item Chose the proposed installation method.
\end{itemize}

The recommended installation mode is \verb|texhash| (\#1). The easiest choice for the destination of \FileName{texmf} is your
home directory (\#2) since you will not need to give your \verb|sudo| password. If you want to follow those recommendations you can directly enter \cmdbox{./setup.sh install texhash default} in your terminal. In that case you will have at the end of the installation a \FileName{texmf} tree in your \EnvVariable{HOME} directory and you can delete the downloaded release and the extracted directory \FileName{AMlatex}. 

\subsection{Installation for Windows}
\label{sec:release:windows}

\begin{itemize}
\item Open the \miktex Settings (admin if possible)
\item Select the Roots tab
\item Add the path to the folder \verb|texmf|. If you work on the department network it is recommended to use the tree located in Software (R:) AMlatex, otherwise use the one you obtained after extraction of the release archive.
\item Refresh the file name database (FNDB) in the General tab.
\end{itemize}

Those instruction can be done via the command line (press \verb|Win+r| and enter \verb|cmd| to open a terminal), for example:
\begin{bashshell}
initexmf --admin --register-root=R:\AMlatex\texmf
initexmf --admin --update-fndb 
\end{bashshell}

\subsection{Installation for Mac OS}
\label{sec:release:mac}

The installation for Mac OS is slightly different from Linux one, since local package are supposed to be installed in \verb|$HOME/Library/texmf| and \cmdbox*{texhash} is not needed. Nevertheless the procedure is the same:
\begin{itemize}
\item Enter \cmdbox{./setup.sh install} in your terminal.
\item Chose method number 1 then valid installation directory.
\end{itemize}

\section{From GitLab}
\label{sec:git}

The project \AMlatex is now hosted in the GitLab of the department at the address 
\url{https://gitlab.lrz.de/AM/AMlatex}. 
If you encounter trouble accessing this repository, please contact the LaTeX or GitLab administrator.
Once access is granted, you just have to enter in your terminal something like:
\begin{bashshell}
cd /path/to/where/you/want/to/clone
git clone git@gitlab.lrz.de:AM/AMlatex.git
\end{bashshell}

%First you need a git repository. 
%It can be easily obtained by cloning the main repository of the department, which is located inside  \url{transfer/03_latex/releases/}. 
%If you intend to share your
%potential modifications and contributions to \AMlatex, you should put your clone somewhere inside
%\url{transfer/02_persistent}. If not, you can save it wherever you want. 

Once you have your own repository, you will have to extract the packages, the classes and the documentation yourself from the sources (the \extension{dtx} files). Again, the script \FileName{setup.sh} will assist you for this. 

\begin{bashshell}
cd /path/to/where/you/want/to/clone
git clone /path/to/LRZ/transfer/03_latex/AMlatex.git
cd AMlatex/
./setup.sh extract sty
./setup.sh extract doc
\end{bashshell}

Note that you will likely encounter some troubles to compile the documentation. Note also that compiling
the documentation is absolutely not mandatory, you can install and use the packages without them.
However if you want to understand the code and modify it, you will need to compile the documentation
successfully. You can find some help in this process in\cite{HOWTOdtx}. It is also a nice opportunity
to test if your \LaTeX{} distribution is up-to-date.


\section{From Common}
\label{sec:common}

In fact this method is exactly the same as \ref{sec:release}, except that you do not need to download a release. On Linux, just have to go in the directory \url{common/Vorlagen/AMlatex/} and run \FileName{setup.sh}, as explained in \ref{sec:release:linux},\ref{sec:release:mac}. That way you will get easily and quickly the last version, the drawback being that this version might be not stable.

On windows, you can use this directory as \miktex-Root as explained in \ref{sec:release:windows}, but you must be aware that if you do that, you will need a working network connection to compile your document, and the compilation will likely be slower as if the packages would be locally installed.

\bigskip
Actually this works with any \FileName{AMlatex} directory that contains a \FileName{setup.sh} file.


\printbibliography

\end{document}






%\item Latex has some standard paths where he looks for packages and classes. One of those is \verb!$HOME/texmf!. Either we copy \texttt{03\_latex/texmf} directly in the home directory or we create a symbolic link to it;
%\begin{lstlisting}
%	cd $HOME
%	ln -s $AMLATEXDIR/03_latex/texmf texmf
%\end{lstlisting}


%\item Clone the git repository \texttt{LRZ\_common/Vorlagen/AMlatex} anywhere you want by using Git Bash or Git Gui. 
%\item If your \LaTeX{} distrubution is not MiKTeX you are crazy enough to know how to install a package. Otherwise open the MiKTeX Settings (admin) and go to the Roots tab. Add the path to the folder texmf (normally it is in \texttt{AMlatex}) and then refresh the file name database (FNDB) in the General tab. 


