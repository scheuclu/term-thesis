\chapter{Rotationspendel} \label{ap:Rotationspendel}



Im Folgenden wird das Beispiel zum rotierenden Pendel aus \cite{Mayet:Tautochronic} dargelegt, 
da die dort hergeleiteten Sachverhalte 
als wichtige Grundlage für die im Rahmen dieser Arbeit durchgeführten Untersuchungen dienen.
Betrachtet wird ein einzelnes Pendel ($n_p = 1$), welches die Möglichkeit hat, im rotorfesten
Koordinatensystem zu rotieren ($\mu_r \neq 0$). 
%
Die geometrische Zwangsbedingung für ein solches Pendel ergibt sich nach \secref{subsec:TautDesignBedingungen} zu
\begin{equation}
		\left(\pdiff{\rho}{s}\right)^2 +  \rho^2 \left(\pdiff{\varphi}{s}\right)^2    + \mu_{r} \left(\pdiff{\alpha}{s}\right)^2  = 1.
				\label{eq:AnhTautochronerAbsorberentwurfGeomZwangsbedRotPendel}
\end{equation}




Werden die Lösungen $\varphi$, für welche kein Rotationsgradient $\pdiff{\alpha}{s}$ auftritt, mit $\varphi_0$ bezeichnet, 
resultiert die Zwangsbedingung für diesen Fall zu
\begin{equation}
		\left(\pdiff{\rho}{s}\right)^2 +  \rho^2 \left(\pdiff{\varphi_0}{s}\right)^2  = 1
				\label{eq:AnhTautochronerAbsorberentwurfGeomZwangsbedRotPendelOhneRot}
\end{equation}
und für $\varphi_0$ gilt mit dem skalierten Radius $\rho^2(s) = 1 - k^2 s^2$ explizit
\begin{equation}
		\left(\pdiff{\varphi_0}{s}\right)^2  = \frac{1}{\rho^2}  \left(   1 - \left(\pdiff{\rho}{s}\right)^2    \right),
				\label{eq:AnhTautochronerAbsorberentwurfGeomZwangsbedDefPhi0}
\end{equation}
wofür eine analytische Lösung existiert, welche in \cite{Mayet:Tautochronic} angegeben ist.
%
%
%
% Geom. Zwangsbedingung für Rotationspendel
%
Mit dieser Definition lautet die geometrische Zwangsbedingung für $\varphi$ für ein rotierendes Pendel
\begin{equation}
		\left(\pdiff{\varphi}{s}\right)^2  =  \left(\pdiff{\varphi_0}{s}\right)^2     -   \frac{\mu_r}{\rho^2} \left(\pdiff{\alpha}{s}\right)^2.   
				\label{eq:AnhTautochronerAbsorberentwurfGeomZwangsbedDefPhi}
\end{equation}
%
%
% Ansatz für Rotationsgradien
%
Der Rotationsgradient wird in \cite{Mayet:Tautochronic} als
\begin{equation}
		\pdiff{\alpha}{s}  = \frac{\rho}{\sqrt{\mu_r}}   \pdiff{\varphi_0}{s} \pdiff{\bar{\alpha}}{s}   
		\qquad \text{mit} \qquad 			  \left| \pdiff{\bar{\alpha}}{s} \right|  < 1 \quad \forall s
				\label{eq:AnhTautochronerAbsorberentwurfAnsatzRotationsgradient}
\end{equation}
%
angesetzt, wobei $\bar{\alpha}(s)$ eine beliebige Funktion von $s$ und deren Betrag kleiner als 1 ist. Damit resultiert
\begin{equation}
		\pdiff{\varphi}{s} =  \pdiff{\varphi_0}{s}  \sqrt{  1 - \left(\pdiff{\bar{\alpha}}{s}\right)^2  },
				\label{eq:AnhTautochronerAbsorberentwurfGeomZwangsbedDefPhiMitSpezAlpha}
\end{equation}
%
wobei jedoch für $\varphi$ keine allgemeine Lösung mehr existiert. Diese wird auch nicht zwingend benötigt, da für das dynamische Verhalten des Systems nur der Rotationsgradient entscheidend ist, wie anhand der Definition von $f(s)$ ersichtlich ist. 
$f(s)$ ergibt sich nach Gleichung 	\eqref{eq:Abkuerzungenfuerf} unter Verwendung der eben dargestellten Definitionen   für ein Rotationspendel zu
%
%
% Defintion von f(s)
%
\begin{equation}
	\begin{split}
		f(s) &= \rho^2 \pdiff{\varphi}{s} + \mu_r \pdiff{\alpha}{s}
				  = \rho^2 \pdiff{\varphi_0}{s}  \sqrt{  1 - \left(\pdiff{\bar{\alpha}}{s}\right)^2  } + \mu_r \frac{\rho}{\sqrt{\mu_r}}   \pdiff{\varphi_0}{s} \pdiff{\bar{\alpha}}{s}  \\
				 &= \pdiff{\varphi_0}{s} \left(\rho^2 \sqrt{  1 - \left(\pdiff{\bar{\alpha}}{s}\right)^2} + \sqrt{\mu_r} \rho  \pdiff{\bar{\alpha}}{s}  \right)	.
				\label{eq:AnhTautochronerAbsorberentwurfDefVonf}
	\end{split}				
\end{equation}
%
In \cite{Mayet:Tautochronic}  wird vorgeschlagen, den Gradienten von $\bar{\alpha}$ %der Einfachheit halber
konstant zu wählen. 
Die Wahl von 
\begin{equation}
		 \pdiff{\bar{\alpha}}{s} = \sqrt{\frac{\mu_r}{1+\mu_r}}
				\label{eq:AnhTautochronerAbsorberentwurfWahlVonAlphaS}
\end{equation}
maximiert die Funktion $f(s)$ an der Stelle $s=0$, was die Wirkung des Absorbers bei kleinen Amplituden erhöht \cite{Mayet:Tautochronic}.
Für die Wahl von  $\pdiff{\bar{\alpha}}{s} = \const$ ergeben sich die Lösungen von $\varphi$ durch Multiplikation von $\varphi_0$
mit einer Konstanten und die resultierende Funktion $f(s)$ lautet 
%
%
% Resultierendes f(s)
%
\begin{equation}
		 f(s) = \sqrt{1+\mu_r} - \frac{k^2}{2 \sqrt{\mu_r+1}} \left(\left(\mu_r + 1\right) k^2 + 1 \right) s^2 +  \mathcal{O}(s^4),
			\label{eq:AnhTautochronerAbsorberentwurfFendgueltig}
\end{equation}
welche für weitere Untersuchungen allgemein als 
\begin{equation}
		 f(s) = b_0 + b_2 s^2 +  \mathcal{O}(s^4) \qquad \text{mit} \qquad b_0, b_2 \in \MR
			\label{eq:AnhTautochronerAbsorberentwurfFAlsPolynom}
\end{equation}
geschrieben wird.


