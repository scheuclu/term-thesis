\chapter{Averaging-Methode} \label{ap:Averaging}



Die \new{Averaging-Methode} oder  \new{Methode der Mittelwertbildung} (engl. \textit{averaging method}) wird zur Analyse 
nichtlinearer Oszillationen eines dynamischen Systems verwendet \cite{Mitropolsky:Averaging}. 
Die Grundidee der Methode ist die Approximation des Ausgangssystems durch ein gemitteltes System, welches dadurch 
deutlich einfacher zu untersuchen ist \cite{Hagedorn:NichtlinSchwingungen1978}.
Aus dem Verständnis der Dynamik des gemittelten Systems  können Rückschlüsse auf die 
Dynamik des Ausgangssystems gezogen werden  \cite{Hagedorn:NichtlinSchwingungen1978}. 
Dabei wird durch Zeitintegration über eine Periode des Ausgangssystems  der Einfluss der hochfrequenten Dynamiken 
herausgefiltert \cite{Hagedorn:NichtlinSchwingungen1978}. 
Die wichtigsten Zusammenhänge zur praktischen Anwendung, wie sie im Rahmen der Herleitungen in dieser Arbeit 
benötigt werden, sind im Folgenden kurz dargelegt, wobei hier die einfachste Form der Mittelwertbildung, 
die \textit{periodische Mittelwertbildung} \cite{Sanders:AveragingMethods2007}, betrachtet wird. 
Die  angegebenen Zusammenhänge geben den Grundgedanken der Averaging-Methode wieder.
Weitere Details, Beweise und Verallgemeinerungen sind in \cite{Sanders:AveragingMethods2007} zu finden. 

Ausgegangen wird von einem System in der Form
\begin{equation}
	\vec{\dot{x}} = \epsilon \vf\left(\vx,t\right) + \mathcal{O}(\epsilon^2), \qquad  \vx\left(0\right) = \va,
\label{eq:AveragingAusgangsgleichung}
\end{equation}
wobei $\epsilon$ ein kleiner Parameter und $\vf$ ein $T$-periodisches Vektorfeld in der Zeit $t$ ist.

Vernachlässigung von Termen höherer Ordnung und Mittelwertbildung durch Integration über $t$ führt auf die \textit{gemittelte Gleichung}
\begin{equation}
	\vec{\dot{z}} = \epsilon \vec{\bar{f}}\left(\vz\right), \qquad  \vz\left(0\right) = \va, 
\label{eq:AveragingGemittelteGleichung}
\end{equation}
%
%
mit
%
%
\begin{equation}
	\vec{\bar{f}}\left(\vz\right) = \frac{1}{T} \int^T_0{\vf\left(\vz,s\right) \dd s} .
\label{eq:AveragingGemittelteGleichungRechteSeite}
\end{equation}
%
%
%
Das grundlegende Ergebnis dieser Mittelung ist, dass die Lösungen des gemittelten Systems nahe (von Ordnung $\epsilon$) 
an den Lösungen des Ausgangssystems in einem Zeitintervall der Ordnung $1/\epsilon$ sind \cite{Sanders:AveragingMethods2007}, was sich als
\begin{equation}
	\left\| \vx\left(t\right) - \vz\left(t\right) \right\| \leq c \epsilon \quad \text{für} \quad 0 \leq t \leq \frac{L}{\epsilon}
\label{eq:AveragingEpsilonNah}
\end{equation}
mit positiven Konstanten $c$ und $L$ schreiben lässt. Unter weiter spezifizierten Anforderungen, 
welche für reale Systeme in der Regel zutreffen, bleiben die Lösungen des gemittelten Systems 
auch für das Zeitintervall $t \in \left[0, \infty \right)$ nahe (von Ordnung $\epsilon$) 
an den Lösungen des Ausgangssystems \cite{Sanders:AveragingMethods2007}.






