\chapter{Zusammenfassung und Ausblick}
\label{cha:Schluss}

Im Rahmen der vorliegenden Masterarbeit wurden 
Untersuchungen zur optimalen Auslegung von Fliehkraftpendeln durchgeführt.


Es wurde allgemein hergeleitet, wie das optimale lineare Tuning $k$
eines Fliehkraftpendels zu bestimmen ist, was nun direkt in den Entwurfsprozess
aufgenommen werden kann.
Untersuchungen des Einflusses eines verallgemeinerten Rotationsgradienten 
und  des Einflusses von  nichtlinearem Mistuning auf das Systemverhalten wurden durchgeführt.
Aus den dabei gewonnenen Erkenntnissen wurden
Aussagen und Vorschläge zur gezielten Beeinflussung und Optimierung
der Fliehkraftpendelwirkung durch spezielle Wahl des Rotationsgradienten
und des Mistunings abgeleitet.




% Zusammenfassung der Punkte
%%%%%%%%%%%%%%%%%%%%%%%%%%%%%%%%%%%%%%%%%%%
% Simulation und Bogenlängenverfahren
Die Untersuchungen wurden auf Basis der Averaging-Gleichungen, welche eine 
Approximation der stationären Systemantwort  darstellen,  durchgeführt.
Da die Systemantwort stark nichtlineares Verhalten mit kritischen Punkten
(horizontale und vertikale Tangenten im Gleichgewichtspfad) aufweist, ist
es nicht möglich diese in geschlossener Form zu lösen.
Um eine systematische, numerische Lösung der Gleichungen zu ermöglichen und etwaige 
Schwierigkeiten klassischer Lösungsverfahren bei kritischen Punkten
zu vermeiden, wurde in dieser Arbeit ein Bogenlängenverfahren implementiert,
welches bei den weiteren Untersuchungen zur Anwendung kam.
Darüber hinaus wurde ein Simulationsprogramm erstellt, 
mit welchem  die Aussagekraft der auf Basis der Averaging-Gleichungen 
prognostizierten  Systemantwort durch Vergleich mit den Ergebnissen 
aus der Simulation der vollständigen, nichtlinearen 
Bewegungsgleichungen des dynamischen Fliehkraftpendelsystems
verifiziert wurde.




% Optimales Tuning
Die Schwingungsreduktionswirkung eines Fliehkraftpendels mit linearem Tuning $k=k_E$ 
ist durch den Einfluss von viskoser Dämpfung  bei der Anregungsordnung $k_E$ nicht optimal.
Das gedämpfte Fliehkraftpendel zeigt nicht bei der Anregungsordnung $k_E$ beste Wirkung,
sondern der Punkt bester Absorberwirkung verschiebt sich zu einer 
anderen Anregungsordnung. 
Durch die Wahl des Absorbertunings $k$ wird eine Veränderung der Geometrie und somit
eine Veränderung der Dynamik erreicht, womit
dem ungewünschten Einfluss der viskosen Dämpfung entgegengewirkt werden kann. 
In dieser Arbeit wurde eine  analytische Bedingung, mit welcher das optimale Tuning $k$ eines
Fliehkraftpendels unter Berücksichtigung linear viskoser Dämpfung bestimmt
und somit direkt im Entwurfsprozess berücksichtigt werden kann, hergeleitet.
%
Damit kann nun  bereits im Auslegungsprozess dem suboptimalen Einfluss der Dämpfung 
vorgebeugt und optimale Absorberwirkung bei der auftretenden 
Anregungsordnung $k_E$ erreicht werden.  
%
Durch den aus der analytischen Bedingung berechneten Wert für das optimale 
Tuning $k$ bleibt die tautochrone Eigenschaft des Fliehkraftpendels erhalten
und es besitzt optimale Schwingungsreduktionswirkung bei der
auftretenden Anregungsordnung. 

 
% Rotationsgradient
In dieser Arbeit wurde der in \cite{Mayet:Tautochronic} als 
konstant vorgeschlagene Gradient $\frac{\partial \bar{\alpha}}{\partial s}$ 
als Reihe angesetzt und damit der Einfluss eines verallgemeinerten Rotationsgradienten 
auf die Absorberwirkung untersucht. 
Auf Basis der Averaging-Gleichungen konnte für die im Rahmen dieser Arbeit getroffenen
Ansätze durch den verallgemeinerten Rotationsgradienten eine scheinbare
Verbesserung der Absorberperformance erzielt werden, welche allerdings durch die
vollständige Simulation nicht bestätigt werden konnte.
Die Averaging-Gleichungen bildeten unter Berücksichtigung des verallgemeinerten Rotationsgradienten das Systemverhalten 
nur in äußerst kleinen Bereichen hinreichend  genau ab.
Auch bei der Erweiterung der getroffenen Ansätze um Terme höherer Ordnung
konnte keine Verbesserung der Absorberperformance %durch einen verallgemeinerten Rotationsgradienten 
erreicht werden. 
Die in dieser Arbeit durchgeführten Untersuchungen bestätigten, dass für ein Rotationspendel
mit den im Rahmen dieser Arbeit zugrunde gelegten  Ansätzen die Wahl von  
$\frac{\partial \bar{\alpha}}{\partial s} = \sqrt{\mu_r/(1+\mu_r)}= \const$,
wie in \cite{Mayet:Tautochronic} vorgeschlagen, wahrscheinlich
optimal ist. 
Diese Erkenntnis wurde bei der weiteren Untersuchung des Einflusses von
nichtlinearem Mistuning berücksichtigt. 

% Untersuchung Mistuning
Durch den gezielten Einsatz von nichtlinearem Mistuning konnte die Robustheit
eines Absorbers  gegenüber Änderungen in der Anregungsordnung deutlich gesteigert werden.
Dabei wird der Anstieg der Absorberamplitude und der  Drehungleichförmigkeit 
zu deutlich größeren Anregungsordnungen verschoben.
Dadurch wurde die Gefahr zum Sprung in der Absorberamplitude, welche für
ein Kreisbahnpendel typisch ist und wozu ein tautochrones Rotationspendel 
auch tendiert, deutlich abgemildert. 
Die drastische Performanceverschlechterung, welche beim Sprung
der Absorberamplitude zwangsweise auftreten würde, wurde damit unterbunden.
Des Weiteren wird beim Einsatz von nichtlinearem Mistuning die Performance 
bei der Anregungsordnung, für welche das Fliehkraftpendel ausgelegt ist,
im Rahmen der Näherung durch die Averaging-Gleichungen nicht beeinflusst.
Der Einfluss des auf Basis der Averaging-Gleichungen ermittelten Vorschlags für das
nichtlineare Mis"-tuning konnte durch die nichtlineare, vollständige Simulation bestätigt werden. 
Da das prognostizierte Systemverhalten durch die Averaging-Gleichungen 
nahezu identisch mit  den Ergebnissen der Simulation ist,
wird das stationäre Absorberverhalten durch die Averaging-Gleichungen sehr gut approximiert.
Die auf Basis der Averaging-Gleichungen unter Berücksichtigung von 
nichtlinearem Mistuning prognostizierte Absorberantwort ist als Auslegungskriterium für den Fliehkraftpendelentwurf
sehr gut geeignet.


Im Rahmen dieser Arbeit konnte das Verhalten mehrerer Fliehkraftpendel
und die damit verbundenen Effekte wie beispielsweise asynchrones Verhalten
nicht untersucht werden. 
Eine mögliche und vielversprechende Anwendung wäre der Einsatz von 
nichtlinearem Mis"-tuning bei Absorbersystemen mit mehreren Freiheitsgraden.
Dabei könnte das nichtlineare Mistuning signifikante Effekte bezüglich
der Vermeidung von asynchronen Pendelbewegungen besitzen.
Zur genauen  Beurteilung solcher Effekte sind aber noch detaillierte Untersuchungen nötig.



