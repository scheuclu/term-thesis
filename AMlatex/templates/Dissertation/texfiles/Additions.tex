%TODO put all this in AMdissertation and create option 'additions'
\documentclass[hidelinks, 12pt, a4paper, fleqn, openright,final]{AMdissertation}

 
%\usepackage{amsmath, amssymb, units}
\usepackage{cleveref}
\setkeys{Gm}{showframe=false} % Draw frame around every page 
%%% DEFINITION FOR CLEVERREF
\crefname{equation}{}{}
\crefname{figure}{Figure}{Figures}
\crefname{table}{Table}{Tables}
\crefname{appendix}{Appendix}{Appendices}
\crefname{chapter}{Chapter}{Chapters}
\crefname{section}{Section}{Sections}
\crefname{subsection}{Subsection}{Subsections}
\crefname{subsubsection}{Subsection}{Subsections}
%%%%%%%%%%%%%%%%%%%%%%%%%%%%%%

\usepackage{AMtikz}

\usepackage[utf8]{inputenc}

\usepackage{tocloft} % For the vspace of the table of contents

\usepackage{currfile} % To get the file name with the \currfilename command (useful for the externalized pgfplots)

\usepackage{import}
%\usepackage{graphicx}
\usepackage{xspace}

\usepackage{booktabs}

%TODO make a package for this purpose
\usepackage[toc,xindy,acronym,nopostdot]{glossaries} 
\usepackage{longtable}
\usepackage{multicol}
\usepackage{glossary-mcols}

% ===== STYLE =======
\makeatletter
\renewcommand*{\@glossarysection}[2]{%
\ifx\@@glossarysecstar\@empty
\else
\@gls@toc{#1}{\@@glossarysec}%
\fi
\@@glossaryseclabel}
\makeatother
% ================

\newacronym{ancf}{ANCF}{absolute nodal coordinate formulation}
\newacronym{ale}{ALE}{arbitrary Lagrangian-Eulerian}
\newacronym{bc}{BC}{basis case}
\newacronym{bosch}{BTT}{Bosch Transmission Technology B.V.}
\newacronym{cog}{COG}{center of gravity}
\newacronym{cpu}{CPU}{central processing unit}
\newacronym{cvt}{CVT}{continuously variable transmission}
\newacronym[longplural={degrees of freedom}]{dof}{DOF}{degree of freedom}  
\newacronym{daf}{DAF}{Doorne Automobiel Fabriek}  
\newacronym[longplural={equations of motion}]{eom}{EOM}{equation of motion}
\newacronym{ein}{EIN}{element impact noise}
\newacronym{fe}{FE}{finite element}
\newacronym{fem}{FEM}{finite element method}
\newacronym{ffr}{FFR}{floating frame of reference}
\newacronym{fft}{FFT}{fast Fourier transformation}
\newacronym{lcp}{LCP}{linear complementarity problem}
\newacronym{ls}{LS}{lower strand}
\newacronym{lrvm}{LRVM}{large rotation vector method}
\newacronym{mbs}{MBS}{multibody system}
\newacronym{mbsim}{MBSim}{multibody simulator}
\newacronym{med}{MED}{medium}
\newacronym{ncp}{NCP}{nonlinear complementarity problem}
\newacronym{nvh}{NVH}{noise, vibration and harshness}
\newacronym{od}{OD}{overdrive}
\newacronym{oep}{OEP}{operational endplay}
\newacronym{pri}{PRI}{primary arc}
\newacronym{re}{RE}{rocking-edge} 
\newacronym{rcm}{RCM}{redundant coordinate method} 
\newacronym{rpm}{RPM}{\ensuremath{\text{revolutions\ per\ minute}}}
\newacronym{sr}{SR}{speed ratio} 
\newacronym{sec}{SEC}{secondary arc}
\newacronym{tap}{TAP}{turnaround point}
\newacronym{tr}{TR}{thrust ratio} 
\newacronym{ud}{UD}{underdrive}
\newacronym{us}{US}{upper strand}
\newacronym{wrt}{wrt.}{with respect to}

\makeglossaries

\newcommand{\ANCF}{\gls{ancf}\xspace}
\newcommand{\ALE}{\gls{ale}\xspace}
\newcommand{\BC}{\gls{bc}\xspace}
\newcommand{\BCs}{\glspl{bc}\xspace}
\newcommand{\Bosch}{\gls{bosch}\xspace}
\newcommand{\COG}{\gls{cog}\xspace}
\newcommand{\CPU}{\gls{cpu}\xspace}
\newcommand{\CVT}{\gls{cvt}\xspace}
\newcommand{\CVTs}{\glspl{cvt}\xspace}
\newcommand{\DAF}{\gls{daf}\xspace}
\newcommand{\DOF}{\gls{dof}\xspace}
\newcommand{\DOFs}{\glspl{dof}\xspace} 
\newcommand{\dofs}{\DOFs} 
\newcommand{\EIN}{\gls{ein}\xspace}
\newcommand{\EOM}{\gls{eom}\xspace}
\newcommand{\EOMs}{\glspl{eom}\xspace}
\newcommand{\eom}{\EOMs}
\newcommand{\FE}{\gls{fe}\xspace}
\newcommand{\FEs}{\glspl{fe}\xspace}
\newcommand{\FEM}{\gls{fem}\xspace}
\newcommand{\FFR}{\gls{ffr}\xspace}
\newcommand{\FFT}{\gls{fft}\xspace}
\newcommand{\LCP}{\gls{lcp}\xspace} 
\newcommand{\LCPs}{\glspl{lcp}\xspace}
\newcommand{\LS}{\gls{ls}\xspace}
\newcommand{\LRVM}{\gls{lrvm}\xspace}
\newcommand{\MBS}{\gls{mbs}\xspace}
\newcommand{\MBSs}{\glspl{mbs}\xspace}
\newcommand{\MBSim}{\gls{mbsim}\xspace}
\newcommand{\MED}{\gls{med}\xspace}
\newcommand{\NCP}{\gls{ncp}\xspace}
\newcommand{\NVH}{\gls{nvh}\xspace}
\newcommand{\OD}{\gls{od}\xspace}
\newcommand{\OEP}{\gls{oep}\xspace}
\newcommand{\PRI}{\gls{pri}\xspace}
\newcommand{\RE}{\gls{re}\xspace}
\newcommand{\RCM}{\gls{rcm}\xspace}
\newcommand{\RPM}{\gls{rpm}\xspace}
\newcommand{\SEC}{\gls{sec}\xspace}
\newcommand{\SR}{\gls{sr}\xspace}
\newcommand{\TAP}{\gls{tap}\xspace}
\newcommand{\TR}{\gls{tr}\xspace}
\newcommand{\UD}{\gls{ud}\xspace}
\newcommand{\US}{\gls{us}\xspace}
\newcommand{\wrt}{\gls{wrt}\xspace}

%% Just shortcuts in the text

\newcommand{\el}{element\xspace}
\newcommand{\El}{Element\xspace}
\newcommand{\els}{elements\xspace}
\newcommand{\Els}{Elements\xspace}
\newcommand{\head}{head\xspace}
\newcommand{\Head}{Head\xspace}
\newcommand{\heads}{heads\xspace}
\newcommand{\Heads}{Heads\xspace}
\newcommand{\PFT}{Push-Force-Tester\xspace}
\newcommand{\pulley}{pulley\xspace}
\newcommand{\Pulley}{Pulley\xspace}
\newcommand{\pulleys}{pulleys\xspace}
\newcommand{\Pulleys}{Pulleys\xspace}
\newcommand{\ring}{ring\xspace}
\newcommand{\Ring}{Ring\xspace}
\newcommand{\rings}{rings\xspace}
\newcommand{\Rings}{Rings\xspace}
\newcommand{\sheave}{sheave\xspace}
\newcommand{\Sheave}{Sheave\xspace}
\newcommand{\sheaves}{sheaves\xspace}
\newcommand{\Sheaves}{Sheaves\xspace}


%% Special Environments
\newtheorem{remark}{Remark}

%% CVT values
%%% Ratios
\newcommand{\iF}{\ensuremath{i_F}\xspace}
\newcommand{\DiF}{\ensuremath{\Delta\iF}\xspace}
\newcommand{\is}{\ensuremath{i_s}\xspace}
\newcommand{\Dis}{\ensuremath{\Delta\is}\xspace}
\newcommand{\ig}[1]{\ensuremath{i_{g#1}}\xspace}
\newcommand{\Dig}[1]{\ensuremath{\Delta\ig{#1}}\xspace}
\newcommand{\eff}{\ensuremath{\eta_{loc}}\xspace}
\newcommand{\effGlob}{\ensuremath{\eta_{glob}}\xspace}
\newcommand{\Deff}{\ensuremath{\Delta\eff}\xspace}
\newcommand{\OERE}{\ensuremath{\upsilon}\xspace}
\newcommand{\DOERE}{\ensuremath{\Delta\OERE}\xspace}


\newcommand{\NEZ}{\ensuremath{N_{E0}}\xspace}



%TODO: das hier müssten eigentlich in sowas wie AMMath
\newcommand{\AIK}{A_{IK}}

% Double-Line-Symbols (mathbb) 
\newcommand{\msymb}[1]{\ensuremath{\mathbb{#1}}}
\newcommand{\msA}{\msymb{A}}
\newcommand{\msB}{\msymb{B}}
\newcommand{\msC}{\msymb{C}}
\newcommand{\msI}{\msymb{I}}
\newcommand{\msM}{\msymb{M}}
\newcommand{\msN}{\msymb{N}}
\newcommand{\msP}{\msymb{P}}
\newcommand{\msS}{\msymb{S}}

% Calligarphy-Symbols
\newcommand{\mc}[1]{\ensuremath{{\cal{#1}}}}
% Definition of the normal cone symbol
\newcommand{\NC}{\mc{N}}
% Energies
\newcommand{\calL}{\mc{L}}
\newcommand{\calV}{\mc{V}}
\newcommand{\calT}{\mc{T}}

%% math symbols for ALE beam
\newcommand{\barx}{\bar{x}}
\newcommand{\rRef}{r _{Ref}}
\newcommand{\tRef}{t _{Ref}}
\newcommand{\nRef}{n _{Ref}}
\newcommand{\bRef}{b _{Ref}}
\newcommand{\qf}[1][]{q_{f #1}}
\newcommand{\dotq}{\dot{q}}
\newcommand{\dotqf}[1][]{\dot{q} _{f#1}}
\newcommand{\dotqk}{\dot{q}_k}

% Derivatives
%% Absolute Derivatives
\newcommand{\differential}{\mathrm{d}}
\newcommand{\difdif}[2]{\frac{\differential #1}{\differential #2}}
\newcommand{\ddifddif}[2]{\frac{\differential^{2} #1}{\differential #2 ^{2}}}
\newcommand{\ddifdifdif}[3]{\frac{\differential^{2} #1}{\differential #2~\differential #3}}
\newcommand{\ddt}[1]{\difdif{#1}{t}}

%% Partial Derivatives
\newcommand{\pd}[2]{\frac{\partial {#1}}{\partial {#2}}}
\newcommand{\pdt}[1]{\pd{#1}{t}}
\newcommand{\pdqk}[1]{\pd{#1}{q_k}}

\usepackage{enumitem}
%\usepackage{caption, subcaption} % For subfigures TODO: subfigures-package required somewhere\ldots why?
%\captionsetup[sub]{format=plain,skip=1ex,position=top,labelfont={bf}}
 %% Format the subcaption design

%TODO this should be a part of AMtikz
\usepackage{tikz}
\usepackage{pgfplots, pgfplotstable}
\usetikzlibrary{external}
\usepgfplotslibrary{groupplots}
\usetikzlibrary{matrix,positioning}
\tikzexternalize[prefix=tikzExternal/,shell escape=-enable-write18]
\tikzexternalize[up to date check=md5]

%\providecommand{\tikzExtPDF}{\tikzset{external/system call={pdflatex \tikzexternalcheckshellescape -shell-escape -halt-on-error -interaction=batchmode -jobname "\image" "\texsource" "\def \tikzexternalrealjob {DOC_140411_Thesis}\input {DOC_140411_Thesis}"}}}
\providecommand{\tikzExtPDF}{\tikzset{external/system call={pdflatex \tikzexternalcheckshellescape -shell-escape -halt-on-error -interaction=batchmode -jobname "\image" "\texsource"}}}
\providecommand{\tikzExtLua}{\tikzset{external/system call={lualatex \tikzexternalcheckshellescape -enable-write18 -halt-on-error -interaction=batchmode -jobname "\image" "\texsource"}}}

\usetikzlibrary{plotmarks}
% and optionally (as of Pgfplots 1.3): 
\pgfplotsset{compat=newest} 
\pgfplotsset{plot coordinates/math parser=false}
\newlength\fheightpgf
\newlength\fwidthpgf

%% Create pgfplot for only legend (from: http://tex.stackexchange.com/questions/131934/adding-a-legend-next-to-subfigures-of-pgfplots)
% argument #1: any options
\newenvironment{customlegend}[1][]{%
    \begingroup
    % inits/clears the lists (which might be populated from previous
    % axes):
    \csname pgfplots@init@cleared@structures\endcsname
    \pgfplotsset{#1}%
}{%
    % draws the legend:
    \csname pgfplots@createlegend\endcsname
    \endgroup
}%

% makes \addlegendimage available (typically only available within an
% axis environment):
\def\addlegendimage{\csname pgfplots@addlegendimage\endcsname}

%% Command to draw vertical line and name it. Additionally (if given) the name could be written at the max or the min position
\newcommand{\specialPos}[3][]{
 \ifstrempty{#1}{%
    \draw [black!40] (axis cs:#2,\pgfkeysvalueof{/pgfplots/ymin}) -- (axis cs:#2,\pgfkeysvalueof{/pgfplots/ymax}) node [below right] {#3};
  }{%
    \draw [black!40] (axis cs:#2,\pgfkeysvalueof{/pgfplots/ymax}) -- (axis cs:#2,\pgfkeysvalueof{/pgfplots/ymin}) node [above right] {#3};
  }%
  }

%% Um das Problem anzugehen, dass 0^x von pgfplots ausgewertet wird. Kopiert von http://tex.stackexchange.com/questions/153978/pgfplots-function-x0-5-skipping-first-sample-sqrt-is-not

\makeatletter
\def\pgfmathfloatpow@#1#2{%
    \begingroup%
    \expandafter\pgfmathfloat@decompose@tok#1\relax\pgfmathfloat@a@S\pgfmathfloat@a@Mtok\pgfmathfloat@a@E
    \ifcase\pgfmathfloat@a@S\relax
        % 0 ^ #2 = 0
        \pgfmathfloatcreate{0}{0.0}{0}%
    \else
        \expandafter\pgfmathfloat@decompose@tok#2\relax\pgfmathfloat@a@S\pgfmathfloat@a@Mtok\pgfmathfloat@a@E
        \ifcase\pgfmathfloat@a@S\relax
            % #1 ^ 0 = 1
            \pgfmathfloatcreate{1}{1.0}{0}%
        \or
            % #2 > 0
            \pgfmathfloatpow@@{#1}{#2}%
        \or
            % #2 < 0
            \pgfmathfloatpow@@{#1}{#2}%
        \or
            % #2 = nan
            \edef\pgfmathresult{#2}%
        \or
            % #2 = inf
            \edef\pgfmathresult{#2}%
        \or
            % #2 = -inf
            \pgfmathfloatcreate{0}{0.0}{0}%
        \fi
    \fi
    \pgfmath@smuggleone\pgfmathresult
    \endgroup
}%
\makeatother 

\usepackage[bordercolor=red!50,backgroundcolor=red!10,linecolor=red!20,textsize=footnotesize,textwidth=20mm]{todonotes}
\setlength{\marginparwidth}{20mm}

%% AVOID that todonotes get externalized
\makeatletter
\renewcommand{\todo}[2][]{\tikzexternaldisable\@todo[#1]{#2}\tikzexternalenable}
\makeatother

%% make it possible to refer to a footnote with a specific label (having basically the same footnote)
\makeatletter
\newcommand\footnoteref[1]{\protected@xdef\@thefnmark{\ref{#1}}\@footnotemark}
\makeatother

%% Enables reading string keys from cvs file
\makeatletter
\pgfplotsset{ 
    /pgfplots/flexible xticklabels from table/.code n args={3}{%
        \pgfplotstableread[#3]{#1}\coordinate@table
        \pgfplotstablegetcolumn{#2}\of{\coordinate@table}\to\pgfplots@xticklabels
        \let\pgfplots@xticklabel=\pgfplots@user@ticklabel@list@x
    }
}
\makeatother

%\if11
%\chairman{\ProfRixen}
%\supervisor{\ProfUlbrich \and Prof. Carlo L. Bottasso, Ph.d.}
%\else
%\chairman{\ }
%\supervisor{\ProfUlbrich \and ?}
%\fi



\usepackage{multicol} % Use two columned stuff (needed e.g. for the glossaries

%% HAck to enable copyright (from: http://tex.stackexchange.com/questions/29895/how-to-get-copyright-when-mixing-t1-fonts-and-fontspec)
%\renewcommand*\copyright{{\usefont{EU1}{lmr}{m}{n}\textcopyright}}

%% Hack to enable nobreakspace (from: http://tex.stackexchange.com/questions/29895/how-to-get-copyright-when-mixing-t1-fonts-and-fontspec)
\DeclareTextCommandDefault{\nobreakspace}{\leavevmode\nobreak\ } 

% Avoid draft mode in hyperref
%\hypersetup{final}


%%%%%%%% SETUP FOR DOCUMENT %%%%%%%%%%%%%%%%%%%%
\author{Kilian Bartimäus Wolfgang Grundl}
\title{Validation of a Pushbelt Variator Model --}
\subtitle{Insights into a non-smooth Multibody System}
\chairman{\ProfRixen}
\supervisor{\ProfUlbrich \and Prof. Carlo L. Bottasso, Ph.d.}
\date{20. Mai 2015}{\hspace{2cm}}
%\dedication{To My Family}

%%%%%%%% END SETUP

% === Bibliography ===
\addbibresource{Meine.bib}
\usepackage{bibentry}

\begin{document}

% ======= LANGUAGE =============
\selectlanguage{ngerman}
% ==============================
%% ==========FONT===============
\sffamily
%  =============================
%% ==========FORMATTING=========
\setlength{\parindent}{0pt}
\setlength{\parskip}{0em}
\pagestyle{empty}
%  =============================

\textbf{\Large Anmerkungen zum Antrag\\[1em]}

Diese Vorlage soll dazu dienen nichts zu vergessen beim Antrag.
Die Graduate School hat dafür eigentlich eine super Seite zusammengestellt:\\
\url{https://www.gs.tum.de/promotion-mit-der-tum-gs/einreichung-der-promotion/}\\
Hier sollen nur noch weitere Anmerkungen / Tips gegeben werden und eine Vorlage zum mit-notieren.

\begin{itemize}
  \item Die beglaubigten Kopien können am Standesamt in Garching (5 EUR) gemacht werden und werden daher von der Uni akzeptiert.\\
  \url{http://www.garching.de/Rathaus+_+Service/Dienstleistungen+_+Lebenslagen/Beglaubigungen.html}\\
  Für TUM-Studenten: Die Bestätigung kann auch direkt am Abgabetermin gemacht werden, falls Urkunde und Zeugnis von der TUM kommen.
  \item Die Bestätigung über die TUM-GS-Mitgliedschaft ist eigentlich eine Bestätigung, dass man das Qualifizierungsprogramm gemacht hat. 
  Diese kann auch nachgereicht werden, da sie erst notwendig ist, wenn die Ukrunde gedruckt wird.
  \item 
\end{itemize}

\newpage

\textbf{\Large Promotionsantrag}
  \begin{multicols}{2}
  \small
  \raggedright
    An die\\
    TECHNISCHE UNIVERSIÄT MÜNCHEN\\
    SSZ Prüfungsamt - Promotionen\\
    Arcisstraße 21\\
    80333 München\\
    Öffnungszeiten: Mo.-Do. 8.30 - 11.30 Uhr\\
    (Zimmer 0186)\\[0.5em]
    \textbf{an die}\\[0.5em]
    \emph{Fakult\"{a}t f\"{u}r Maschinenwesen}\\[0.5em]
    Erwerb des Doktorgrades:\\
    Doktor-Ingenieur (Dr.-Ing.)\\[1em]
    \underline{\textbf{Anlagen gemäß \S\,8 Promotionsordnung}}\\[1em]
    Nachweis über geforderte Vorbildung:\\
      a) Diplom- oder Masterurkunde\\
      b) Zeugnis über das Staatsexamen, oder Ärztliche / Zahnärztliche Prüfung\\
      \textbf{a+b in beglaubigter Kopie}\\
      ggf. Anerkennung des ausländischen Studiums\\[0.5em]
      5 gleichlautende Exemplare der Dissertation (DIN A4) \textbf{mit Titelblatt nach Anlage 4} der Promotionsordnung und \textbf{fest gebunden (Leimbindung)}\\[0.5em]
      Zusammenfassung für das Jahrbuch mit Unterschrift des Doktoranden und des 1. Prüfers \textbf{(s. https://www.ub.tum.de/publizieren\-dissertation)}\\[0.5em]
      Erklärung nach Anl. 5 der Promotionsordnung\\[0.5em]
      Auflistung der Vorveröffentlichungen\\[0.5em]
      Lebenslauf, \textbf{insbes. Bildungsgang (deutsch)}\\[0.5em]
      Amtliches Führungszeugnis \textbf{(von Behörde an Behörde und nicht älter als 3 Monate)}\\
      {\centering oder}\\
      gültigen Arbeitsvertrag mit der TU München \textbf{(Orig. + Kopie oder beglaubigte Fotokopie)}\\[0.5em]
      Promotionsstudium an der TU München?\\
      wenn ja, Matrikel-Nr.\,\,02870484\\[0.5em]
      Mitglied der TUM Graduate School?\\
     (wenn ja, bitte Bestätigung beifügen)\\
     $\boxtimes$\,ja\qquad $\square$\,nein\\[0.5em]
     Bestätigung über den Eintrag in die Promotionsliste
     \vfill\columnbreak
     % =========== 2nd Page ============
     Garching, \today\\[0.5em]
     Ich überreiche hiermit eine wissenschaftliche Abhandlung über:\\[1em]
     \emph{\DissTitle}\\[1em]
     und bitte, diese mit den neben aufgeführten Anlagen an die\\[1em]
     \emph{Fakult\"{a}t f\"{u}r Maschinenwesen}\\[1em]
     zur Weiterbearbeitung im Sinne der Promotionsordnung zuzuleiten.\\[1.5em]
     \underline{\textbf{Persönliche Angaben}}\\[0.5em]
     Vor- und Zunamen:\\
     XXX XXX\\[0.5em]
     Geburtstag und -ort:\\
     XX. XX 19XX in XXX\\[0.5em]
     Akademischer Grad: Dipl.-Ing. (Univ.)\\[0.5em]
     Berufsbezeichnung:\\
     Ingenieur\\[0.3em]
     Staatsangehörigkeit: deutsch\\[2em]
     \raisebox{-1em}{$\overline{\makebox[0.45\textwidth]{\centering\footnotesize(Unterschrift)}}$}\\[1.5em]
     Anschrift der Wohnung mit Tel.-Nr.:\\
     XXX XX\\
     D-XX XXX\\
     +49.XX.XXXXXX\\[0.5em]
     Heimatanschrift mit Tel.-Nr.:\\
     (keine weitere) \\[0.5em]
     Tagsüber erreichbar unter Tel.-Nr.: \\
     +49.XX.XXXXXX\\[0.5em]
     E-mail: XXXX@XX.XX
  \end{multicols}

\pagebreak
\textbf{\Large Zusammenfassung für das Jahrbuch\\[1em]}

\selectlanguage{english}
\textbf{\large English}\\[1em]
\textbf{\DissTitle}\\
\emph{\DissSubTitle}\\[1em]
The work treats the validation of a non-smooth multibody model for continuously variable transmissions (CVTs) using a pushbelt.
Based on available approaches, models concerning the dynamics, the internal contacts as well as the numerics are enhanced and partly newly formulated.
Influences due to numerical and physical parameters are studied.
The applicability of the full model for stationary and instationary operational states is shown.\\[2em]

\selectlanguage{ngerman}
\textbf{\large Deutsch}\\[1em]
\textbf{Validierung eines stufenlosen Schubgliederband-Variatormodells}\\
\emph{Einblicke in ein nichtglattes Mehrkörpersystem}\\[1em]
Die Arbeit behandelt die Validierung eines nichtglatten Mehrkörpermodells von stufenlosen Schub\-gliederband-Variatoren (CVTs).
Auf vorliegenden Ansätzen aufbauend werden Dynamikmodelle, Kontaktmodelle sowie deren Numerik verbessert und teilweise neu formuliert.
Weiterhin erfolgt eine Untersuchung der numerischen und physikalischen Parameter.
Die Eignung des vollständigen Modells für stationäre sowie instationäre Betriebszustände wird gezeigt.\\\hrule
\vspace{2cm}

Daten auf dem online protokoll:\\
(Online muss diese Zusammenfassung in Deutsch und Englisch eingetragen werden. Man kann sie zwischenspeichern und bekommt einen key, bzw. eine URL, die -- merkt man sich diese -- es ermöglicht den Text / den Titel o.Ä. noch einmal zu bearbeiten.)
\textbf{ACHTUNG}: Kopiert man den Text aus dem Latexdokument werden uU nicht nur zusätzliche Bindestriche mit kopiert, sondern auch Zeilenumbrüche, die dann wieder in der von der TUM erstellten PDF auftauchen und zu Fehlern führen können. 

\vspace{3em}
\raisebox{-1em}{$\overline{\makebox[0.45\textwidth]{\centering\footnotesize(Kilian Grundl)}}$} \hfill \raisebox{-1em}{$\overline{\makebox[0.45\textwidth]{\centering\footnotesize(\ProfUlbrich)}}$}

% == VERÖFFENTLICHUNGEN ==
\printbibliography[title=Vorveröffentlichungen]
\nocite{*}
\vspace{3em}
  Garching, \today\hfill%
  \raisebox{-1em}{$\overline{\makebox[90mm]{\centering\footnotesize(Unterschrift)}}$}%
\vfill
\thispagestyle{empty}
\pagebreak
% ========================

\textbf{\Large Eidesstattliche Erkl\"{a}rung\\[1em]}
Ich erkl\"{a}re an Eides statt, dass ich die bei der
\begin{center} \emph{Fakult\"{a}t f\"{u}r Maschinenwesen} \end{center}
  der TUM zur Promotionspr\"{u}fung vorgelegte Arbeit mit dem Titel:
  \begin{center}
    \emph{\DissTitle}
  \end{center}
  \vspace*{-1ex}
  am
  \vspace*{-1ex}
  \begin{center}
    \emph{Lehrstuhl f\"{u}r Angewandte Mechanik, Fakult\"{a}t f\"{u}r Maschinenwesen}
  \end{center}
  unter der Anleitung und Betreuung durch
  \begin{center}
    \emph{XXX}
  \end{center}
  ohne sonstige Hilfe erstellt und bei der Abfassung nur die gem\"{a}\ss{} \S\,6 Abs.~6 und~7 Satz~2 angegebenen Hilfsmittel benutzt habe.\par
  \begin{itemize}
    \renewcommand*\labelitemi{$\square$}%
    \setlength{\itemsep}{0.2\itemsep}%
    % \setlength{\parskip}{0.5\parskip}%
    % \setlength\topsep{0pt}%
    % \setlength\partopsep{0pt}%
    \item[$\boxtimes$] Ich habe keine Organisation eingeschaltet, die gegen Entgelt Betreuerinnen und Betreuer f\"{u}r die Anfertigung von Dissertationen sucht, oder die mir obliegenden Pflichten hinsichtlich der Pr\"{u}fungsleistungen f\"{u}r mich ganz oder teilweise erledigt.
    \item[$\boxtimes$] Ich habe die Dissertation in dieser oder \"{a}hnlicher Form in keinem anderen Pr\"{u}fungsverfahren als Pr\"{u}fungsleistung vorgelegt.
    \item Die vollst\"{a}ndige Dissertation wurde in
    %\ifx\Affirm@publishedInString\@empty%
    \hrulefill\par
    %\else
    %  \textnormal{\Affirm@publishedInString}
    %\fi
    % \hrulefill\par
    % \hrulefill\par
    ver\"{o}ffentlicht.
    Die promotionsf\"{u}hrende Einrichtung Fakult\"{a}t f\"{u}r Maschinenwesen hat der Vorver\"{o}ffentlichung zugestimmt.
    \item[$\boxtimes$] Ich habe den angestrebten Doktorgrad \emph{noch nicht} erworben und bin \emph{nicht} in einem fr\"{u}heren Promotionsverfahren f\"{u}r den angestrebten Doktorgrad endg\"{u}ltig gescheitert.
    \item Ich habe bereits am \rule{2cm}{0.4pt}\ bei der Fakult\"{a}t f\"{u}r \hrulefill\par
    der Hochschule \hrulefill\
    unter Vorlage einer Dissertation\par mit dem Thema \hrulefill\par
    % \hrulefill\par
    % \hrulefill\par
    die Zulassung zur Promotion beantragt mit dem Ergebnis:
    \hrulefill%\par\hrulefill
  \end{itemize}
  %
  Die \"{o}ffentlich zug\"{a}ngliche Promotionsordnung der TUM ist mir bekannt, insbesondere habe ich die Bedeutung von \S\,28 (Nichtigkeit der Promotion) und \S\,29 (Entzug des Doktorgrades) zur Kenntnis genommen.
  Ich bin mir der Konsequenzen einer falschen Eidesstattlichen Erkl\"{a}rung bewusst.
  
  Mit der Aufnahme meiner personenbezogenen Daten in die Alumni-Datei bei der TUM bin ich
  \begin{itemize}
    \renewcommand*\labelitemi{$\square$}%
    \setlength{\itemsep}{0.2\itemsep}%
    \item[$\boxtimes$] einverstanden
    \item[$\square$] nicht einverstanden.
  \end{itemize}
  \vfill
  Garching, \today\hfill%
  \raisebox{-1em}{$\overline{\makebox[90mm]{\centering\footnotesize(Unterschrift)}}$}%

\end{document}
