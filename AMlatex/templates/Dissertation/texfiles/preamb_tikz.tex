%TODO this should be a part of AMtikz
\usepackage{tikz}
\usepackage{pgfplots, pgfplotstable}
\usetikzlibrary{external}
\usepgfplotslibrary{groupplots}
\usetikzlibrary{matrix,positioning}
\tikzexternalize[prefix=tikzExternal/,shell escape=-enable-write18]
\tikzexternalize[up to date check=md5]

%\providecommand{\tikzExtPDF}{\tikzset{external/system call={pdflatex \tikzexternalcheckshellescape -shell-escape -halt-on-error -interaction=batchmode -jobname "\image" "\texsource" "\def \tikzexternalrealjob {DOC_140411_Thesis}\input {DOC_140411_Thesis}"}}}
\providecommand{\tikzExtPDF}{\tikzset{external/system call={pdflatex \tikzexternalcheckshellescape -shell-escape -halt-on-error -interaction=batchmode -jobname "\image" "\texsource"}}}
\providecommand{\tikzExtLua}{\tikzset{external/system call={lualatex \tikzexternalcheckshellescape -enable-write18 -halt-on-error -interaction=batchmode -jobname "\image" "\texsource"}}}

\usetikzlibrary{plotmarks}
% and optionally (as of Pgfplots 1.3): 
\pgfplotsset{compat=newest} 
\pgfplotsset{plot coordinates/math parser=false}
\newlength\fheightpgf
\newlength\fwidthpgf

%% Create pgfplot for only legend (from: http://tex.stackexchange.com/questions/131934/adding-a-legend-next-to-subfigures-of-pgfplots)
% argument #1: any options
\newenvironment{customlegend}[1][]{%
    \begingroup
    % inits/clears the lists (which might be populated from previous
    % axes):
    \csname pgfplots@init@cleared@structures\endcsname
    \pgfplotsset{#1}%
}{%
    % draws the legend:
    \csname pgfplots@createlegend\endcsname
    \endgroup
}%

% makes \addlegendimage available (typically only available within an
% axis environment):
\def\addlegendimage{\csname pgfplots@addlegendimage\endcsname}

%% Command to draw vertical line and name it. Additionally (if given) the name could be written at the max or the min position
\newcommand{\specialPos}[3][]{
 \ifstrempty{#1}{%
    \draw [black!40] (axis cs:#2,\pgfkeysvalueof{/pgfplots/ymin}) -- (axis cs:#2,\pgfkeysvalueof{/pgfplots/ymax}) node [below right] {#3};
  }{%
    \draw [black!40] (axis cs:#2,\pgfkeysvalueof{/pgfplots/ymax}) -- (axis cs:#2,\pgfkeysvalueof{/pgfplots/ymin}) node [above right] {#3};
  }%
  }

%% Um das Problem anzugehen, dass 0^x von pgfplots ausgewertet wird. Kopiert von http://tex.stackexchange.com/questions/153978/pgfplots-function-x0-5-skipping-first-sample-sqrt-is-not

\makeatletter
\def\pgfmathfloatpow@#1#2{%
    \begingroup%
    \expandafter\pgfmathfloat@decompose@tok#1\relax\pgfmathfloat@a@S\pgfmathfloat@a@Mtok\pgfmathfloat@a@E
    \ifcase\pgfmathfloat@a@S\relax
        % 0 ^ #2 = 0
        \pgfmathfloatcreate{0}{0.0}{0}%
    \else
        \expandafter\pgfmathfloat@decompose@tok#2\relax\pgfmathfloat@a@S\pgfmathfloat@a@Mtok\pgfmathfloat@a@E
        \ifcase\pgfmathfloat@a@S\relax
            % #1 ^ 0 = 1
            \pgfmathfloatcreate{1}{1.0}{0}%
        \or
            % #2 > 0
            \pgfmathfloatpow@@{#1}{#2}%
        \or
            % #2 < 0
            \pgfmathfloatpow@@{#1}{#2}%
        \or
            % #2 = nan
            \edef\pgfmathresult{#2}%
        \or
            % #2 = inf
            \edef\pgfmathresult{#2}%
        \or
            % #2 = -inf
            \pgfmathfloatcreate{0}{0.0}{0}%
        \fi
    \fi
    \pgfmath@smuggleone\pgfmathresult
    \endgroup
}%
\makeatother 

\usepackage[bordercolor=red!50,backgroundcolor=red!10,linecolor=red!20,textsize=footnotesize,textwidth=20mm]{todonotes}
\setlength{\marginparwidth}{20mm}

%% AVOID that todonotes get externalized
\makeatletter
\renewcommand{\todo}[2][]{\tikzexternaldisable\@todo[#1]{#2}\tikzexternalenable}
\makeatother

%% make it possible to refer to a footnote with a specific label (having basically the same footnote)
\makeatletter
\newcommand\footnoteref[1]{\protected@xdef\@thefnmark{\ref{#1}}\@footnotemark}
\makeatother

%% Enables reading string keys from cvs file
\makeatletter
\pgfplotsset{ 
    /pgfplots/flexible xticklabels from table/.code n args={3}{%
        \pgfplotstableread[#3]{#1}\coordinate@table
        \pgfplotstablegetcolumn{#2}\of{\coordinate@table}\to\pgfplots@xticklabels
        \let\pgfplots@xticklabel=\pgfplots@user@ticklabel@list@x
    }
}
\makeatother