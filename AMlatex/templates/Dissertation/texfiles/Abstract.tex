\section*{Abstract}

Continuously variable transmissions (CVTs) enable optimal control of the drivetrain and thereby reduce fuel consumption.
In addition they increase comfort in stop-and-go drive states.
CVTs are therefore an ideal solution for modern car concepts.\\
Previous work developed detailed nonlinear multibody models of its system's dynamic.
This thesis discusses the enhancement and validation of these pushbelt CVT models.\\
For the efficient simulation of the whole model the dynamics of its parts, their interactions and the numerical treatment are discussed.
New models for the ring dynamics and for the contact between curved sheaves and curved element flanks, an identification method for the element's longitudinal contact stiffness, a physical based tracking law as well as methods for computing time reduction improve and enhance the system.\\
The discussion of numerical aspects covers the number of simulated elements, the usage of non-smooth dynamics and a comparison of different ring models.
The chapter physical aspects covers parameter uncertainties, especially the contact parameters for geometry, stiffness and friction.
A sensitivity analysis of the system is performed.
It yields the identification of these parameters.
Furthermore, optimization possibilities are gained.\\
Local forces between elements, rings and pulleys correlate on a qualitative and quantitative level with measurements. 
The kinematics in the arcs in terms of spiral running and element pitch angle fit to available data as well.
Furthermore, simulation results yield global values, i.e. thrust ratio, efficiency, slip-curves and an endplay tolerance, that match to measurements.\\
Overall the validation demonstrates the applicability of the model for well identified parameters for stationary and instationary phenomena and its fit -- qualitatively as well as quantitatively.
\newpage
%TODO AMlang
%\selectlanguage{ngerman}
\section*{Zusammenfassung}
Stufenlose Getriebe (CVTs) ermöglichen die optimale Regelung des Antriebsstranges und tragen damit zur Reduktion von Treibstoffen bei.
Weiterhin erhöhen sie den Komfort im Stop-and-go-Verkehr.
CVTs sind daher eine ideale Lösung für moderne Automobilkonzepte.\\
Frühere Arbeiten entwickelten ein detailliertes nichtlineares Mehrkörpermodel der Systemdynamik eines Schubgliederbandgetriebes.
Diese Dissertation diskutiert die Weiterentwicklung sowie die Validierung dieses Modells.\\
Zur effizienten Simulation des gesamten Systems wird die Dynamik der einzelnen Komponenten, deren Interaktionen sowie die numerische Behandlung diskutiert.
Neue Modelle der Ringdynamik und für den Kontakt zwischen balligen Scheiben und balligen Elementflanken, eine Methode zur Identifizierung der Längs\-steifigkeit der Elemente, ein physikalisch begründetes \glqq Tracking\grqq-Gesetz sowie verschiedene Methoden zur Reduzierung der Berechnungszeiten verbessern und erweitern das Gesamtsystem.\\
Die Diskussion numerischer Aspekte des Modells beinhaltet die Anzahl der simulierten Elemente, die Anwendung nicht-glatter Kontaktgesetze und den Vergleich verschiedener Ringmodelle.
Das Kapitel \glqq Physikalische Aspekte\grqq\xspace behandelt Parameterungenauigkeiten -- insbesondere die der Kontaktparameter für Geometrie, Steifigkeit und Reibung.
Die Sensitivität des Modells gegenüber diesen wird untersucht.
Diese Untersuchung trägt zur Identifikation der Parameter bei und zeigt Optimierungsmöglichkeiten auf.\\
Lokale Kräfte aus der Simulation zwischen Elementen, Ringen und Scheiben stimmen qualitativ und quantitativ mit Messungen überein.
Die Kinematik in den Bögen bezüglich dem Spirallauf und der Elementneigung passen zu vorhandenen Messdaten.
Weiterhin werden aus den Simulationen globale Daten gewonnen.
Das Kraftverhältnis, die Effizienz, \glqq Schlupfkurven\grqq\xspace und die Toleranz gegenüber Spiel im Band gleichen einander in Simulation und Messung.
Insgesamt zeigt die Validierung die Anwendbarkeit des Modells für stationäre wie instationäre Lastfälle.
Es können qualitative sowie quantitative Aussagen über das System getroffen werden, die sich in Experimenten bestätigen lassen. 
\vfill
\selectlanguage{english}

