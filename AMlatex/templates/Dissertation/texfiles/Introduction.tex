\chapter{Introduction} \label{chap:Introduction}
% Einleitung: Erst Überblick über das was es gibt
% - Arbeit einordnen
% -- Überblick über dieses Kapitel

% Typical approach: specific modeling -> stationary modeling: This approach is more general to check assumptions of simpler models, but also enable broader basis for further research / optimization
% Modeling aspects:    Broader basis for studies (e.g. Curved Sheaves), enhancements concering sub-parts (Ring model, Simulation Time)
% Validating aspects:  How good are the models --> reference model for further investigations (Companies work with small models, but this is highly nonlinear and therefore there is a high demand for a relyable detailed model of the pushbelt CVT)

In 1958 the first \DAF-car with a \emph{Variomatic} was released.
Its rubber belt was replaced with a steel pushbelt in the 1970s and the system was then called \emph{Transmatic} which later became the \CVT.
In 1985 mass production started and the pushbelt has been enhanced over the years.\\
The share of \CVTs on the transmission market will increase from 6 \% in 2012 to up to 14 \% in 2020 \cite{idler_fresstragfahigkeit_2014}.
The continuously ratio coverage enables to operate the engine in an optimal point.
In combination with a downsizing of the engine the reduction of fuel consumption in the overall drivetrain can be reached.
According to \cite{srivastava_modeling_2006} passenger cars create about 10 \% of the world wide greenhouse gases.
The \CVT-system can therefore help to reduce the $CO_2$ footprint.
The increasing urbanization \cite{martine_state_2007} will lead to more stop-and-go drive states in modern cities.
A \CVT avoids discrete shifting, which improves the comfort of the driver.\\
This analysis shows that the pushbelt \CVT is an ideal solution for modern car concepts.
A further improvement of the system can increase its efficiency and robustness.
Recent research optimizes the system control as well as the tribology.
Reliable models, which enable deeper insights into the system, are necessary. 
They form the basis for design-optimization.\\
Various models cover particular aspects of the pushbelt \CVT dynamics.
Restrictive assumptions limit their scope.
No all-encompassing model has been derived and validated.
The correct representation of the full nonlinear dynamics for transient maneuvers covering spatial motion was the goal of a joint research project between the Institute of Applied Mechanics of the Technische Universität München and \Bosch\footnote{formerly ``Van Doorne Transmissie``}.
Preceding PhD-theses \cite{bullinger_dynamik_2005, geier_dynamics_2007, schindler_spatial_2010, cebulla_spatial_2014} form the basis for this contribution.
The pushbelt \CVT is represented as a \MBS modeling the dynamics and interactions of its parts separately.
Detailed output of the simulations allow for deep analysis of the inner kinematics and kinetics.\\
Each model demands for validation to prove its significance.
This work concerns itself with the analysis of the chosen approach.
The single submodels are discussed of which some are enhanced.
Furthermore, the overall models are analyzed in different aspects.
Due to the approximation of the models with computer technology, the numerical influences are a crucial factor.
Parameter uncertainties limit the range of informative value of the model.
A sensitivity study yields this range.
It furthermore demonstrates the analysis possibilities of the model.
On the one hand the inner mechanics can be analyzed.
On the other hand global measures and transient dynamics can be used for future optimization.
For this goal, the validation of the model is necessary.\par

This chapter introduces the geometric setup and the working principle of a pushbelt \CVT.
A current research overview follows together with literature remarks concerning \MBS.
The most important equations are given.
Prior work, on which this thesis relies on, is finally covered in more detail.
The overview about the complete thesis concludes this chapter.

\section{Geometric Setup and Working Principle} \label{sec:GeometricSetup}

\begin{figure}[h!tp]
\begin{center}
  \includegraphics[width=0.49\textwidth]{1180_Pushbelt_RGB}
  \hfill
  \includegraphics[width=0.49\textwidth]{30_er_Schubgliederb_Gruppe2}
  \caption[Pushbelt CVT]{Pushbelt CVT \copyright\Bosch. It shows the two pulleys -- each with a loose and a fixed sheave -- and the pushbelt consisting of two \rings and about 400 steel \els.}
  \label{fig:PIC_141223_Bosch-Variator-Pushbelt-Elements}
\end{center}
\end{figure}

This work analyzes the dynamics of a pushbelt \CVT as it is depicted in \cref{fig:PIC_141223_Bosch-Variator-Pushbelt-Elements}.
In the following the geometric setup is described.
Thus, the main working principles are explained using basic modeling ideas of \MBSs.
%Different notions for the single parts, e.g. \emph{Elements} \cite{schindler_spatial_2010,cebulla_spatial_2014}, \emph{Blocks} \cite{kanehara_study_1994} or \emph{Segments} \cite{micklem_belt_1994} and \emph{Rings} \cite{schindler_spatial_2010, cebulla_spatial_2014}  or \emph{Bands} \cite{micklem_belt_1994}, are used in the literature. 
The notions valid for this thesis are introduced to serve as a basis for a general and clear description.\\
Besides other works, e.g. \cite{dittrich_theorie_1953, gerbert_influence_1985, bullinger_dynamik_2005}, the understanding of the working principle can be supported by three dimensional visualizations, e.g. videos or animations.
These type of media cannot be included here.
Two examples -- besides many more which can be found in the world wide web -- are:
\begin{itemize}
  \item Animation of \Bosch, the manufacturer of the pushbelt \cite{bosch_mobility_solutions_en_2013}
  \item Animation of DrivelineNEWS.com \cite{drivelinenews_push_2015}
\end{itemize}
\par
%
\begin{figure}[h!tp]
\begin{center}
  \def\svgwidth{\textwidth}
  \import{../multimedia/}{PIC_140921_CVTOverview.pdf_tex}
  \caption[Overview of the pushbelt CVT]{Overview of the pushbelt \CVT with the most important external and internal forces}
  \label{fig:PIC_140921_CVTOverview}
\end{center}
\end{figure}
%
\Cref{fig:PIC_140921_CVTOverview} shows an overview about the general kinematics of the system with the most important external and internal forces.
A torque $M$ is transmitted from a \emph{primary} pulley (Index $P$), i.e. the driving side, to a \emph{secondary} pulley (Index $S$), i.e. the driven side.
The pulleys have the axle distance $d_A$.
The respective wrapped angles are called \PRI and \SEC.
A pushbelt is clamped in both pulleys.
It consists of about 400 \els that are guided by two \ring-packages or \rings, each consisting of about 6 - 12 layers.\\
The ratio between the angular speeds $\omega_S$ of the secondary and $\omega_P$ of the primary pulley 
%
\begin{align}
  \label{eq:speedratio}
  i_S = \frac{\omega_S}{\omega_P}
\end{align}
%
is called \SR.
The \UD configuration is defined by $i_S < 1$ and the \OD configuration by $i_S > 1$.
Ratios for $i_S \approx 1$ are referred as \MED configuration.\\
The \SR is controlled by the ratio of the two clamping forces, i.e. the \TR,
%
\begin{align}
  \label{eq:ThrustRatio}
  i_F \defined \frac{F_P}{F_S}
\end{align}
%
and due to the V-shape of the pulley sheaves and the \el flanks respectively.\\
A geometric ratio
%
\begin{align}
  \label{eq:GeometricRatioGeneral}
  i_{g*} \defined \frac{r_{P*}}{r_{S*}}
\end{align}
%
is defined as the ratio between the \PRI-radius $r_{P*}$ and the \SEC-radius $r_{S*}$.
Different measures for the radii are possible using either \ring or \el kinematics. 
On each pulley the forces are applied due to hydraulic pressure on the respective loose sheave, which can move along the shaft.
The other sheave of the pulley is fixed on the shaft.\par
%
Along the belt four main parts are separated.
Within the \US (Index $u$) and \LS (Index $l$) the forces are transmitted with two dominating effects. 
On the one hand push-forces ($F_{Pu}$ and $F_{Pl}$) are active between the \els.
On the other hand tension forces within the \rings ($F_{Tu}$ and $F_{Tl}$) contribute to the overall equilibrium.
The push-forces between the \els do contribute in the torque transmission in case of $F_{Pu} > F_{Pl}$, which is the intended behavior.
In case of $F_{Pu} < F_{Pl}$, the push-forces work against the torque as the difference in ring-tensions contribute more than the applied torque.
In case of $F_{Pu} = F_{Pl}$ the \emph{transition torque} is applied to the system.
The strand with the higher push-force is called push-strand.
Otherwise it is referred to as the loose-strand.\\
If no push-forces are acting in one of the strands an \OEP exists.
It is defined as the sum of all (positive) gaps between the \els.
It affects the dynamic behavior of the system and depends on the assembly of the belt as well as the applied loads.\\
In the two arcs the torques are transmitted via frictional forces.
These act between \els and sheaves as well as between \els and \rings.
For the default configuration the \els run in from a loose-strand, i.e. the \LS in this case.
Thus, the \els transmit low (or no) push-forces at the first part of the \PRI -- the so called \emph{idle arc}.
Push-forces rise in the \emph{active arc} of the \PRI after the \TAP,.
These push-forces are reduced to the level of the loose-strand in the \SEC.
However -- depending on the configuration\footnote{i.e. boundary conditions, contact positions, stiffness as well as friction values} -- the forces can increase in the first part of the arc yielding another \TAP.\par
%
\begin{figure}
  \centering
    \def\svgwidth{0.6\textwidth}\import{../multimedia/}{PIC_130905_Element_Overview_Directions.pdf_tex}
    \caption{The \el with local directions}
    \label{fig:PIC_130905_Element_Overview_Directions}
\end{figure}
%
For the force transmission the \els are the key part.
All contact forces are transmitted via the \els. 
\cref{fig:PIC_130905_Element_Overview_Directions} shows a sketch with the contact positions.
The torque is transferred into the belt at both flanks on the left and on the right side.
It is distributed on the one hand via contact forces among the \els themselves either at the \RE, the \head or the pin.
On the other hand the contacts on the saddle or the ear introduce tension forces in the \rings\footnote{which can carry up to 40\% of the transmitted torque \cite{srivastava_review_2009}}. 
\par

Three local directions are defined for the \el.
The \emph{longitudinal} (sometimes also \emph{tangential}) direction points from the back to the front of the \el. 
The \emph{radial} direction points from the bottom of the \el to the \head.
The third direction is defined as the \emph{axial} direction from the left flank of the \el to the right flank. 
It is therefore also (disregarding small rotations of the \el) the axial direction of the axes in the \PRI and the \SEC.\\
With the three local directions three rotation directions are defined which follow the standard definition in automotive or aeronautics.
\emph{Pitching} is the rotation around the local axial direction, \emph{rolling} around the longitudinal direction and the \el \emph{yaws} around the radial direction.

\section{State of the Art} \label{sec:StateOfTheArt}
% - Aufzeigen, was schon gemacht wurde auf dem Gebiet (allgemein), welche Lücken vorhanden sind, was Sinn dieser Arbeit ist --> Hinweis, dass sich in den einzelnen Kapiteln weitere Hinweise ergeben werden, da sich das so als Stückwerk sammelt

% Was wurde bisher allgemein auf dem Gebiet gemacht, wie ist der Überblick, wo sind die Lücken
In the last decades different research groups and authors have contributed to the development and enhancements of \CVTs.
This development is driven by both, measurements of the system and its modeling.
This section gives the state of the art with literature focusing on models of pushbelt \CVTs.
Models concerned with the stationary behavior and the ones focusing on the shifting dynamics are separated.
It emphasizes that two different approaches exist in the literature to model these two different modes of the \CVT.
Finally, current research topics are presented.\\
The model of this thesis combines both modes into one detailed model.
Non-smooth multibody dynamics is utilized for it.
Thus, the well-established theory in this field is introduced.
The section concludes with a detailed summary of prior work, on which this thesis is based on.

\subsection{Pushbelt CVT} \label{subsec:ThePushbeltCVT}

A detailed overview about the dynamic modeling of \CVTs and thus a good starting point for further literature research show \textsc{Srivastava et. al.} (2009) \cite{srivastava_review_2009}. 
In the following, a short chronological overview is given pointing out the most important works concerning pushbelt \CVTs.
Thereby stationary models and those who analyze the shifting dynamics are separated.
The most recent research is given at the end.

\subsubsection{Stationary Models}

\textsc{Dittrich} (1953) compares in \cite{dittrich_theorie_1953} experimental results of conical belt drives to theoretical calculations based on the theory of \textsc{Eytelwein} (1808) \cite{eytelwein_handbuch_1808}.
He identifies the effect of spiral running and discusses the necessity of over-clamping that leads to a loss of efficiency \cite{simonis_theorie_1959}. 
He recognizes the complexity of the system and thus draws mainly qualitative conclusions.

\textsc{Gerbert} (1985) discusses the pushbelt mechanics in a planar steady-state model focusing on the velocity distribution of the \els and \rings.
The model uses  Coloumb friction, omits inertia of the \els and regards an uncoupled plate deformation.
Low and high torque states are distinguished and their influence on the push-forces as well as on the arc mechanics, i.e. idle and active arc, are discussed.
The results show the basic effects in a pushbelt \CVT that can be found in measurements -- especially concerning the separation of \rings and \els. 

\textsc{Van Roij} and \textsc{Schaerlaeckens} (1993) discuss the force equilibria within the pushbelt \cite{van_rooij_krafte_1993, van_rooij_krafte_1993-2, van_rooij_krafte_1993-1}.
Three separated models are derived to compute the clamping forces.
The \UD configuration before transition torque is treated as well.
Based on the resulting force distribution the effects on internal losses are given.\par

\textsc{Fuij}, \textsc{Kurokawa}, \textsc{Kanehara} and \textsc{Kitagawa} (1993 - 1995) did a remarkable effort in measurements for the \CVT variator which are published in four parts \cite{fujii_study_1993, fujii_study_1993-1, kanehara_study_1994, kitagawa_study_1995}.
Besides global analysis of thrust- , torque- and speed-ratio, local measurements have been carried out.
They yield the forces acting between the \els, \rings and sheaves and give a qualitative reference.
Stationary and shifting cases are measured and evaluated.\par

\textsc{Sattler} (1999) develops in his thesis a planar \CVT model with a continuous belt for stationary situations.
The focus is set on the dynamic situation in the two arcs omitting the dynamics in the strands. 
Besides a parameter variation the work validates the model with measurements concerning global output values like thrust ratio and efficiency \cite{sattler_stationares_1999}.\\ 
All in all the model is well suited for chains. 
However, certain effects in the pushbelt, e.g. separation of the \els or the interaction of the \els with the \rings, are missing.\\ 
\textsc{Sue} (2003) enhances the model of \textsc{Sattler} to enable faster calculations.
He analyzes the effect of different sheaves on the efficiency of the system \cite{sue_betriebsverhalten_2003}.\par

\textsc{Srnik} (1999) develops the first fully dynamic model of a chain-type \CVT \cite{srnik_dynamik_1999}.
He uses the theory of non-smooth \MBS. 
It considers transient dynamics as well as oscillatory effects.
The separated modeling of the single pins in the chain identifies the polygonal effect to be a important driver for oscillations in the system.
A detailed deformation representation of the sheaves with the pins yields a good correlation with measurements.\\ 
\textsc{Sedlmayer} (2003) enhances the model of \textsc{Srnik} using a spatial representation.
He shows, that the misalignment of the pulleys leads to higher internal forces in the chain and analyzes optimization strategies to reduce these.\par

\textsc{Shimizu et. al.} (2000) model the pushbelt \CVT as a spatial \FE system, however quantitative or qualitative results are missing \cite{shimizu_development_2000}.\\
\textsc{Saito et. al.} (2002-2011) choose a similar approach as in \cite{shimizu_development_2000}.
They use \FEs for the single layers of the \rings, rigid or flexible \els and rigid sheaves comprising a deflection motion \cite{saito_development_2002, saito_application_2006, saito_study_2011}.
Simulation times comparable to ``the [\ldots] time as [\ldots] for durability testing'' \cite[p. 228]{saito_application_2006} yield promising results.
Yet, it stays unclear which contact forces and geometries are used and which model assumptions are made to gain the results. 
An analysis of the stresses within the \rings shows the influence of misalignment and suggests improvement possibilities in design.
Different stiffness as well as geometry changes are analyzed concerning local stresses within the parts and the efficiency of the full variator.\par  

%\textsc{Akehurst} (2001) investigates on the loss mechanisms within pushbelt \CVTs \cite{akehurst_invesitgation_2001}. 
% \todo{Beschäftigt sich auch mit losses, paper von 2004 vorhanden, ist aber nicht so wichtig für unsere Forschung}

\textsc{Bullinger} (2005) models a pushbelt \CVT as a planar but fully dynamic \MBS. 
He remarks that \els should be modeled separated \cite[p. 82]{bullinger_dynamik_2005}. 
Due to a lack of appropriate algorithms, he chooses a quasi-continuous model of the pushbelt covering the single \ring layers as well as the single \els.
Using an Eulerian description for the kinematics he is able to model stationary cases efficiently and to validate his results with measurements from the literature.\par 

\textsc{Lebrecht} (2007) derives a simplified model to enable the coupling to drivetrain models and fast analysis of transient dynamics. 
He correlates the results to the models of \textsc{Srnik, Sedlmayer} and \textsc{Bullinger} to measurements.
Furthermore, he compares chain and pushbelt \CVTs in terms of efficiency and force distribution.
Finally, he analyzes the \NVH properties of both types and points out the influence of Stribeck friction on the \NVH behavior.% and finds a minimal model to explain scratch noise. 

\textsc{Geier} (2007) develops a planar dynamic \CVT model.
He separates the dynamics of the \els and the \rings \cite{geier_dynamics_2007} and enables transient dynamics. 
Based on this \textsc{Schindler} (2010) derives a spatial pushbelt \CVT model using \MBS \cite{schindler_spatial_2010, schindler_spatial_2012}.
\textsc{Cebulla} (2014) enhances the model of \textsc{Schindler} in \cite{cebulla_spatial_2014}.
This work is based on these models.
Detailed summaries are covered in \cref{subsec:PriorWork}. 

\subsubsection{Shifting analysis}

\textsc{Sun} (1988) develops on basis of results of \textsc{Gerbert} a planar but transient model of the pushbelt \CVT assuming one dimensional continuous \els and \ring-layers \cite{sun_performance_1988}. 
Neglecting the pulley and \ring deformation he is able to derive a set of implicit equations and considers ratio- as well as torque changes.
He discusses his results qualitatively.\par

\textsc{Ide} (1995) analyzes the shift behavior and builds a heuristic model based on measurements~\cite{ide_simulation_1995}.
\textsc{Carbone et. al.} (2005 - 2007) develop a theoretical model for the transient dynamics of chain or pushbelt \CVTs, which represents the \emph{creep mode} as well as the \emph{slip mode}~\cite{carbone_influence_2005, carbone_cvt_2007}. The one-dimensional model using an in-elastic band with kinematical prescribed spiral running shows good correlation with performed measurements.
I is successfully used in further applications \cite{simons_shift_2008}.\par

\textsc{Srivastava} (2005 - 2007) develops a model for the transient dynamics of a continuous one-dimensional metal V-pushbelt \CVT \cite{srivastava_transient_2005, srivastava_transient_2007}. Focusing on high-speed applications the influence of inertia effects is treated neglecting belt deformation but incorporating sheave deformations kinematical.

\subsubsection{Recent Research}

Whereas the improvement of models for pushbelt \CVTs has slowed down in the recent years, research treats more the overall behavior of the system.
The focus is set on reducing the safety factor for control and increasing the efficiency.\\
\textsc{Van der Sluis et. al.} (2013) gives an overview about the state of the art for pushbelt \CVT development together with current research focus \cite{van_der_sluis_key_2013}.
Based on ``Key Performance Indicators'', i.e. efficiency and comfort, he points out the latest improvement strategies.
Higher friction coefficients between \els and sheaves as well as enhanced control to reduce safety reduce losses.
The identification of noise sources helps to improve the comfort. 

\textsc{Idler} (2014) employs in his thesis the scuffing for \CVT variators using pushbelts and chains in combination with different variator systems \cite{idler_fresstragfahigkeit_2014}.
He enhances the models of \textsc{Sattler} and \textsc{Sue} as well as of \textsc{Tenberge} concerning shifting with experimental data for the spiral running and the bearing forces.
Furthermore, he develops a formula for the traction coefficient to respect the normal force influence on the friction.
Based on a model for the local surface temperature, he enables the prediction of scuffing in \CVTs.

%\textsc{Shabrov et. al.} (2014) \todo{Der kommt nicht rein, der macht nur Unsinn!}

\textsc{Ji et. al.} (2014) are interested in the differences of the dynamic behavior of the system for micro- and macro slip states \cite{ji_power_2014}.
With a simple model, which neglects deformation effects, they suggest a mechanism to identify the state the variator online by using external vibrations.
The comparison to measurements yield qualitative correlation.\par

\textsc{Lee et. al.} (2014) suggest a logic to optimize the clamping forces and thus enhance efficiency \cite{lee_study_2014}.
With a simple force based model tests on a hardware in the loop and a field vehicle show improved results.\par

%\textsc{Zhang et. al.} (2014) are interested in the pulley strain \cite{zhang_research_2014}. 
%A simple model is developed whereas it stays unclear which assumptions are used.\par
%\textsc{Wang et. al.} (2014)  \cite{wang_study_2014}.\todo{Advanced Materials Research (Volume 952), 28\$, kann man noch nicht bekommen --> vielleicht bei Hr Böhm anrufen, 28456}

\textsc{Van der Noll et. al.} (2015) investigate on the \NVH behavior of the variator \cite{van_der_noll_gerauschoptimierung_2015}. 
They correlate the \EIN phenomenon with eigenmodes of the pulley sheaves and the running-in frequency of the \els in the belt.
Optimization strategies suggest a randomized \el order.
Experiments are performed successfully.

\subsection{Multibody Dynamics} \label{subsec:MultibodyDynamics}

% Was ist das, Geschichte, Lage von heute (was ist Forschungsfeld, wo findet es Anwendung)
% Welche Vorarbeiten gibt es auf denen ich ganz konkret aufbauen kann (Arbeit, Programmierung, modelbildung und Implementierung) --> 
% Wo starte ich, warum habe ich das gemacht, welche Lücke soll geschlossen werden?
% Im Bereich Pushbelt CVTs am Institut, welche Software nutze ich
% Generelle Formel Beschreiben, wie ist MBSim aufgebaut -> nicht glatt, flexibel, generelles framework
As discussed in the previous section, different models of \CVTs exist. 
Most of them are limited in one or more aspects.
Some effects are neglected, e.g. misalignment, transient dynamics or pulley deformation.
For a deep understanding of the complete system, it is necessary to model the full dynamics of a pushbelt \CVT.
This thesis uses the theory of \MBS including nonlinear and non-smooth dynamic effects.  
An introduction to the theory of \MBS is given in this section such that the models discussed in the following chapters can be explained.
As the implementation is done in the software \textsc{MBSim} the theoretical discussion is based on its idea of \MBSs \cite{forg_mbsim_2015, schindler_analysing_2010}.\par

\subsubsection{Literature Remarks}

Many authors discuss the treatment of systems as a \MBS.
Different textbooks are available that show the derivation of the \EOMs for smooth or non-smooth modeling.
It is not possible to list all works.
The following examples for different approaches may serve the reader as a reference.
1997 \textsc{Schiehlen} and \textsc{Shabana} summarized in two articles the state of rigid and flexible \MBSs referencing more than 200 other works \cite{schiehlen_multibody_1997,shabana_flexible_1997}.

An overview about the whole field of smooth dynamics is given in \cite{shabana_dynamics_2005}.
\textsc{Shabana} introduces the basic concepts of dynamical systems to set up a \MBS. 
The \EOMs for rigid bodies with bilateral constrained conditions are derived.
A special focus is set on deformable bodies using the \FFR-formulation or the \ANCF-method to represent nonlinear deformation fields of deformable bodies.

\textsc{Glocker} discusses in his thesis rigid body systems with friction and impacts theoretically \cite{glocker_dynamik_1995}. 
He introduces the fundamental formulations for unilateral and bilateral non-smooth contacts with tangential laws, e.g. friction, and derives the \EOMs, which are then measure differential equations.
He treats coupled and partly elastic impacts.
An even broader basis for the theory and numerical treatment of non-smooth system can be found in \cite{acary_numerical_2008}.
\textsc{Acary} and \textsc{Brogliato} treat non-smooth phenomena also in electrical circuits.

\textsc{Förg} develops a numerical framework for general, i.e. spatial and non-smooth \MBSs \cite{forg_mehrkorpersysteme_2007}. 
He derives the \EOMs with the constrained conditions and shows numerically efficient solution algorithms using proximal functions which yield a nonlinear equation system.
\textsc{Zander} treats flexible \MBSs with non-smooth effects on the basis of \textsc{Förg} \cite{zander_flexible_2009}.
He points out the necessity of locality for the discretization of a flexible body when treating impacts.

\subsubsection{Theory}

\begin{figure}
\begin{center}
  \def\svgwidth{\textwidth}\import{../multimedia/}{PIC_150201_MBS_General.pdf_tex}
  \caption[Overview of a MBS]{Overview of a \MBS with three bodies connected with joints or contacts and excited with forces or prescribed motion.}
  \label{fig:PIC_150201_MBS_General}
\end{center}
\end{figure}

%In the following the equations of a non-smooth dynamical system -- as implemented in \MBSim -- are summarized.
\paragraph{Dynamics}
A \MBS consists of $n$ rigid or deformable bodies as sketched in \cref{fig:PIC_150201_MBS_General}.
The kinematics of every body $i$ is defined by a mathematical model with a set of \DOFs which are the generalized positions $q^i$ and the generalized velocities $u^i$.
Both can be summarized in vectors for the whole system as
%
\begin{align}
  \label{eq:GeneralizedPositionsAndVelocities}
  q^T &= \bmat{q^1 & \ldots & q^n} & u^T &= \bmat{u^1 & \ldots & u^n} & z^T &= \bmat{q^T  & u^T} 
\end{align}
%
where $z$ denotes the state vector of the \MBS.\\
Forces $\lambda$ or impulses $\Lambda$ couple the motion and define the change of acceleration or velocity of the bodies respectively.
Generally two different kinds of coupling laws can be chosen for joints, contacts, external forces and prescribed movements.\\
Single-valued laws define the reaction force explicitly\footnote{For practical examples \cite[p. 18]{forg_mehrkorpersysteme_2007}} from the state vector and the time $t$.
%
\begin{align}
  \label{eq:SingleValuedReactions}
  \lambda = \lambda(z,t)
\end{align}
%
No impulses are covered.\\
Set-valued laws constrain the motion by a condition\footnote{e.g. in normal contact direction a non-negative distance between the contact partners is enforced}.
This induces, that the system may change in a non-smooth way, i.e. there are jumps in the velocity.
Impacts, i.e. impulses, have to be taken into account as well with
%
\begin{align}
  \label{eq:SetValuedReactions}
  \left(\lambda,\Lambda,z,t\right) \in \NC
\end{align}
%
where \NC summarizes all set-valued conditions of the system which have to be fulfilled by the state, the forces and impacts at every time.\\
The \EOMs for the whole non-smooth system are then a measure differential equation with
%
\begin{subequations}
  \label{eq:EOMsNonSmooth}
\begin{align}
  \label{eq:EOMsNonSmoothPos}
  &\dot{q} = T u\\
  \label{eq:EOMsNonSmoothVelSmooth}
  &M \dot{u} = h + W \lambda\\
  \label{eq:EOMsNonSmoothVelNonSmooth}
  &M_k (u^+_k - u^-_k) = W_k \Lambda_k\\
  \label{eq:EOMsNonSmoothNC}
  &\left(\lambda,\Lambda_k,z,t\right) \in \NC
\end{align}
\end{subequations}
%
where $T = T(q)$ is the transformation between generalized velocities and positions, $M = M(q)$ the mass matrix of the system and $h(z,t)$ the smooth right-hand side vector. 
The wrench matrix $W = W(q)$ defines the direction on how forces or impacts, which result from a set-valued reaction, project into the directions of the generalized velocities.\\ 
Equation \eqref{eq:EOMsNonSmoothVelSmooth} shows the smooth velocity change for the system whereas \eqref{eq:EOMsNonSmoothVelNonSmooth} contains the non-smooth velocity changes for times $t_k$, the times of an impact within the system.\par

\paragraph{Integration}
For a numerical integration of \eqref{eq:EOMsNonSmooth} it is necessary to discretize the \EOMs in time. 
The half explicit time-stepping scheme on velocity level of \cite[section 4.4.1]{forg_mehrkorpersysteme_2007} has proven most sufficient for the present model.
It is summarized shortly in the following.\\
Starting from time $t_i$ with the state $z^i$ one approximates with the time step size $\Delta t^i$ the generalized positions of the next time step with
%
\begin{align}
  \label{eq:IntegrationOfGeneralizedPositions}
  q^{i+1} &\approx q^i + T(q^i) u^{i} \Delta t^i &\Rightarrow \tilde{z}^{i} &= \bmat{q^{i+1}\\ u^i}
\end{align}
%
yielding an intermediate state $\tilde{z}^{i}$ which is used for the evaluation of the following equations.\\
The constraints for the continuous motion \eqref{eq:EOMsNonSmoothVelSmooth} and non-continuous motion \eqref{eq:EOMsNonSmoothVelNonSmooth} can be treated simultaneously as continuous forces can be interpreted as an impact for a finite time step.
The velocities of the next time step are approximated with $\tilde{z}^{i}$.
%
\begin{align}
  \label{eq:IntegrationOfGeneralizedVelocities}
  u^{i+1} \approx u^i + M^{-1}(q^{i+1}) \left( h(\tilde{z}^{i},t^{i+1}) \Delta t^i + W(q^{i+1}) \Lambda^{i+1}\right)
\end{align}
%
The relative velocities $\gamma$ that are associated with the impulses $\Lambda$ depend on the generalized velocities and can be written as
%
\begin{align}
  \label{eq:IntegrationOfRelativeVelocities}
  \gamma^{i+1} = (W^{i+1})^T u^{i+1} + w^{i+1}
\end{align}
%
where $w^{i+1}$ is a time dependent vector\footnote{Normal impacts with an elastic part are not of interest here but discussed in \cite{glocker_dynamik_1995, forg_mehrkorpersysteme_2007}.}.
Inserting \eqref{eq:IntegrationOfGeneralizedVelocities} in \eqref{eq:IntegrationOfRelativeVelocities} leads to a linear equation between the impulses $\Lambda$ and the relative velocities $\gamma$ that are in general complementary to each other.
The mass-action-matrix\footnote{also called ``Delassus-Matrix''} is defined to be
%
\begin{align}
  \label{eq:MassActionMatrix}
  G = W^T M^{-1} W
\end{align}
%
which plays a central role for the non-smooth solution\footnote{The time step is not given as it is a general definition for time-stepping algorithms}.
For this time-stepping algorithm $G$ is symmetric.\\
Projection functions reformulate the complementarity problem of equations \eqref{eq:IntegrationOfGeneralizedVelocities} and \eqref{eq:IntegrationOfRelativeVelocities} to a nonlinear system.
%
\begin{align}
  \label{eq:NonlinearFunctionUsingProx}
  F(\Lambda^{i+1}, \gamma^{i+1}) = 0
\end{align}
%
\subsection{Prior Works}
\label{subsec:PriorWork}

Three works, based subsequently on each other, are the direct background for this thesis. 
All are summarized here in detail emphasizing the most important parts. 
\textsc{Geier} modeled a planar pushbelt CVT using independent kinematics of the \els. 
\textsc{Schindler} enhanced the model using spatial dynamics for all parts of the system. 
\textsc{Cebulla} finally focused on implementing new force laws and optimizing the kinematical description.

\subsubsection{Dynamics of Push Belt CVTs}

% First to model all major effects: cooperating: single \els, beam with deflection and elongation (testen von stäben --> nicht sinnvoll), flexible sheaves, transient movement --> also validating this qualitatively with measurments fomr Honda
% however: planar, standard / simple, geometry, not modular 

The works introduced in \cref{subsec:ThePushbeltCVT} model specific effects of the \CVT system. 
Depending on the application some major effects are omitted that are important for a detailed analysis of a pushbelt \CVT. 
\textsc{Geier} is the first to incorporate all main phenomena within a planar system in \emph{Dynamics of Push Belt CVTs} \cite{geier_dynamics_2007}.\par

The \els can move independently from the \rings and are only coupled by physically based contact laws.
Sixteen contact points are defined on one \el which could be evaluated fast as they are mostly based on analytical kinematics.
For the contact between \el and \ring a minimal distance has to be found with a nonlinear function.
The \rings are modeled with the \RCM using co-rotated-\FEs. 
Each \FE has an internal set of \DOFs, which describes a nonlinear rigid body movement in a floating reference frame and in addition local deformations. 
Besides deflection, also elongation of the \rings is possible with this approach.
The \FEs are coupled due to a global coordinate set. 
For the planar case the transformation function is linear between both coordinate sets.
The reader is advised to \cite{zander_flexible_2009} for further details.
Besides the detailed modeling of the \rings, also a quasi-statical deformation of the pulley set is included.
Spiral running is therefore represented well.
All submodels are formulated in a time independent manner such that transient behavior could be studied.\par

In the model the number of \els could be chosen freely. 
Based on 160 \els that are used in the simulation a number of about 650 % ~ 160*3+32*5
\DOFs and about 1440 % ~ 160*9 (16 contact points accoring to \cite[p.45]{Geier2007}, but only 9 are used!)
contact points result\footnote{Up to 1500 \DOFs and 3500 contacts are possible in case of 400 simulated \els \cite[p.81]{geier_dynamics_2007}.}.
The results are compared qualitatively to literature values. % \cite{fujii_study_1993, fujii_study_1993-1, fujii_study_1993-2}. 
Local forces and kinematics could be validated as well as global trends for stationary conditions.
Further simulation results show qualitatively good results.
The model therefore has proven successful for the simulation of a pushbelt \CVT disregarding spatial effects.

\subsubsection{Spatial Dynamics of Pushbelt CVTs}
\label{subsub:SpatialDynamicsOfPushbeltCVTs}

\textsc{Schindler} covered also spatial effects in \emph{Spatial Dynamics of Pushbelt CVTs} \cite{schindler_analysing_2010}. 
The \els are modeled dynamically as rigid bodies with six \DOFs.
Elastic deformations are covered in the contact laws.
The \el-\el interaction is modeled as a nonlinear function.
The \el-sheave interaction uses a linear penalty function.
Tilting deformation of the sheaves is constrained by linear force laws to cover the coupled deformation of the flat pulley sheaves and the shafts.
Flexible sheaves based on a Reissner-Mindlin plate theory using \FE are developed but no results are given.
Two \rings are simulated using an extended \RCM including out of plane movements.
In the spatial case of the \RCM model the internal \FE and the global coordinate sets are coupled in a nonlinear system.
It is solved during each time step in the simulation \cite[p. 23]{schindler_spatial_2010}.
The \els are bound bilaterally to the \rings in normal direction with a set-valued contact law.
Compared to the planar case the contact kinematics gets more complex.
Spatial conditions have to be covered leading to more search directions along the surfaces.
Due to the increased number of \DOFs of about 1200 combined with around 2100 contact points\footnote{based on 160 simulated \els}, %160*(4+2+2+1+4) = 2080 (flanks, Rings, RE, Head, Pin-Hole)% 
the simulation costs increased\footnote{Up to 2650 \DOFs and 5200 contacts are possible in case of 400 simulated \els.}.
To reduce computational time, a stationary belt model based on \cite{sattler_stationares_1999} is derived. 
The transition time at the beginning of each simulation could be shortened compared to a full run up.
Parallelization techniques are tested for single simulations.
A parallel computation of different load cases has been found to be more efficient.% \cite[p.70]{Schindler2010}
\par

Results are compared to values of literature and to measurements conducted by \Bosch.
All in all correct trends are presented for global as well as for local values.
However, improvements are suggested concerning the \ring model, the \el-pulley interaction and possible model order reduction techniques. 
The work offers a basis for a deeper analysis of spatial effects within a pushbelt \CVT.

% Enhancements concerning spatial movement: enhanced ring model, full spatial bodies, dynamic tilting of sheaves, enhanced initialization, more flexibility due to general system,
% no sheave deformation, long simulation times (ring-\el interaction)

\subsubsection{Spatial Dynamics of Pushbelt CVTs: Model Enhancements to a Non-smooth Flexible Multibody System}

% enhancements concerning physical movement, ring-tracking model (?), 
\textsc{Cebulla} focused on \CPU time reduction to further improve the models described in \cref{subsub:SpatialDynamicsOfPushbeltCVTs}.
Due to the complex contact kinematics and the high number of \DOFs, the key part of the model concerning \CPU time are the \rings.
A spatial rod model, which is presented in \cite{lang_multi-body_2011}, follows the theory of \textsc{Cosserat} and uses finite differences for the discretization,
It shows promising results and therefore is adapted for the usage within the \CVT model.
The planar version is described in \emph{Spatial Dynamics of Pushbelt CVTs: Model Enhancements to a Non-smooth Flexible Multibody System} \cite{cebulla_spatial_2014}.
A planar version of the pushbelt \CVT is therefore set up to test and validate additions to the model.\\
A coupled contact law for the quasi-static representation of the pulley-sheave deformation in the spatial case has been derived and added to the model. 
For the solution of the resulting \LCP a \textsc{Lemke} algorithm or a reformulated system using proximal functions is suggested.
While the same coupling technique of the sheaves is used as in \cite{geier_dynamics_2007} different coupling options for the \els are presented.
Yet, these have not been validated.\\
The \el-\ring contact has been enhanced in two steps.
At first the bilateral contact in normal direction at the \el saddle has been replaced by two unilateral contacts at saddle and ear.
This yields a remarkable \CPU time reduction \cite[p.129]{cebulla_spatial_2014}.
In a second step, the penalty function for the axial movement of the \rings on the saddle has been changed from a stiff bilateral function to a heuristic model for the so called ring-tracking.
The effect of this model has not been tested.
To cover production uncertainties, tapered \els have been added to the model. 
An adapted initialization process is given to reduce the transition phase to a stationary point in the simulation.
Different settings of the boundary conditions are enabled and the necessary equations are derived theoretically.
Simulation results are compared to values from literature and measurements from \Bosch.
\par

Altogether \textsc{Cebulla} extended the model of \textsc{Schindler} in many points and shows improvements of different submodels.
Yet, a detailed analysis of the model is missing. 
The present work discusses the described models with different submodels in detail \wrt their numerical and physical behavior.
New submodels have also been added where the analysis showed necessity for improvement.

\subsubsection{Summary of the model}

The three works above lead consecutively to a complex and detailed model of a metal V-pushbelt \CVT.
The most important properties and assumptions are summarized here in a compact manner. 

\begin{itemize}
  \item Spatial effects, e.g. misalignment, are modeled besides a planar version.
  \item Transient dynamics, e.g. torque or speed run up situations as well as shifting movement, are covered
  \item The \els are treated as single, i.e. non-continuous, rigid bodies covering nonlinear deformation due to contact-laws quasi-statically.
  \item The \sheaves are treated as single rigid bodies with a coupled deformation, leading to e.g. spiral running.
  \item The \rings are modeled as one (planar model) or two (spatial model) flexible bodies regarding elongation, in- and out-of-plane bending along the belt.
  \item Contacts, i.e. normal forces and friction forces, are modeled either non-smooth or regularized possibly using nonlinear functions, e.g. \el deformation or Stribeck friction.
  \item No overall kinematical assumptions are made concerning contact situations in the arcs, e.g. separation of active or idle ar or separation of the \els in the loose-strand. 
  Yet, the effects result out of dynamic simulation.
\end{itemize}

\section{Structure of the Work}
%
%\begin{figure}
%\begin{center}
%  \def\svgwidth{\textwidth}\import{../multimedia/}{PIC_140418_Thesis_Overview_V02.pdf_tex}
%  \caption{Overview of the thesis}
%  \label{fig:PIC_140418_Thesis_Overview}
%\end{center}
%\end{figure}
%
An overview about this work is visualized in \cref{fig:PIC_140418_Thesis_Overview}. 
After motivating the thesis the system pushbelt \CVT is described in \cref{sec:GeometricSetup} to introduce the most important concepts to the reader.
The current research state is given in \cref{sec:StateOfTheArt}.
A literature overview for pushbelt \CVTs shows the state of modeling at the time of writing.
Models exist to investigate on specific settings of the \CVT.
Yet, a detailed model covering all major effects is still missing.
To enable such a model the method of \MBS is chosen.
The well-established theory is summarized. 
A detailed model of this system has been set up and enhanced in the past which is described at last.
It forms the starting point for this work. 
\par

The validation of models is an iterative process.
Comparing to measurements combined  with qualitative discussion of the physical interpretable behavior of a first simple model yield improvement ideas for the next one.
Eventually a model should consider all relevant effects being as simple as possible to be interpretable after all \cite{ulbrich_maschinendynamik_1996}.
The iterative process is in fact a loop in which every part affects the others leading to new ideas for either modeling, simulating or validating.
The sequence for this thesis starts with the modeling in \cref{chap:Modeling}, then discusses the sensitivity on numerical and physical parameters in \cref{chap:NumericalAspects,chap:PhysicalAspects} respectively to finally validate the state of the model in \cref{chap:Validation}.
It is emphasized that the results of this work can only be seen in context of all these chapters.
\par

Following this introduction the dynamical models are discussed in \cref{chap:Modeling}.
Three aspects are discussed separately, i.e. the dynamics of the single parts, their interaction and the resulting numerical challenges.\\
Efficient models demand rigid dynamics of the \els and \sheaves respecting their deformation in interactions.
The \rings however deform nonlinear dynamically.
A specific model covering the overall motion with a minimal number of \DOFs is derived, which enables fast integration.
Selective adaption of the dynamical representation, e.g. spatial motion or spiral running, allows for efficient integration within specific application.\\
Interaction models are crucial for the correct representation of the force equilibrium.
The adaption of the \el's thickness leads to a varied stiffness which is respected in the contacts.
The contact kinematics for curved surfaces of \sheaves and \els are derived to enable geometrical adaptions.
The coupled deformation of the two \pulley-\sheaves due to the contact to all \els in one arc is suggested furthermore.
The nonlinear deformation of the \els in longitudinal direction is respected with a special force law.
An updating process using a special measuring device, i.e. the \PFT, is performed to gain precise parameters 
The spatial interaction between \el-saddle and \ring is analyzed in means of \ring-tracking. 
Former models are discussed and a new one is suggested, which respects the normal force influence.\\
The computation of the models on a \CPU leads to numerical issues, which are treated at last.
A kinetics based initialization scheme shortens the transition phase and thus the simulation time.
The application of modern solution schemes for the complementarity problem between \els and \sheaves reduces the simulation time further in every time step.
Finally, the usage of sparse matrix algebra in combination with the specific properties of the integration scheme yields faster integration when non-smooth force laws are used in the model.\\
The post-processing steps conclude the chapter which serve as basis for the further analysis of different aspects of the system.\par

Models are tested on their qualitative output as well as on their sensitivity on parameters.
Numerical aspects are treated in \cref{chap:NumericalAspects}.
To save simulation time less \els are used in simulation than in reality.
The influence of this is studied at first to find a minimal number of \els and to show the difference to a reference simulation, i.e. the overall quality.
Then the influence of different contact laws is studied using either single- or set-valued functions which affects both the simulation time as well as the physical output behavior.
Finally different dynamical models for the \rings are compared in means of \CPU costs and physical output.
A further analysis tests the applicability of the approach for spatial simulation.\par

Besides numerical based parameters, also physical based parameters are not known exactly.
Their impact on the system's behavior is discussed in \cref{chap:PhysicalAspects}.
This forms the basis for two improvements.
On the one hand the parameters can be updated for precise outputs of the model.
On the other hand it offers optimization strategies for future belt designs.\\
The number of \els that are used within the belt assembly is not known precisely.
The parameter is set by experience.
However, small changes lead to different system outputs which are treated in \cref{sec:SimulationElements}.\\
In \cref{sec:ElementPulleyContact} the most important parameters for the contact between \el-flank and \pulley-\sheave are discussed.
The contact position along the flank is unknown as well as the stiffness in axial direction.
Both influence the system's behavior strongly.\\
The same holds for the contact in-between the \els.
The geometry for the \head depends on the element specification.
Thus the contact point is uncertain in terms of radial as well as longitudinal position.
Both aspects are discussed in \cref{sec:ElementElementContact}.\\
Finally, the friction parameters for the contact between \el-\pulley and between \el-\ring are analyzed in \cref{sec:Friction}.\par

The model is validated in \cref{chap:Validation} based on the discussion of the previous chapters.
Force distributions along the belt and kinematics in the arcs are compared qualitatively for a broad spectrum.
Furthermore, quantitative measurements are compared for the most important local forces.\\
Global measures, i.e. descriptive values for certain stationary cases, are compared for varying boundary conditions.
The trends for the thrust ratio and efficiency yield quantitative matches for a variator with well identified parameters.
Other variator setups show the necessity for an updating process.
Furthermore, the influence on external loads and the robustness against e.g. wear are validated with measurements.\\
At last, instationary effects are treated.
On the one hand the gear-rattle noise can be related virtually to the frictional curves.
On the other hand shifting situations are compared to values from the literature qualitatively.\par

Finally, the work is summarized in \cref{chap:Conclusion}.
Based on the conclusion an outlook on further research possibilities is given.
