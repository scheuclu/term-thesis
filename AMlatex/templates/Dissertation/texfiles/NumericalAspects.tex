\chapter{Numerical Aspects} \label{chap:NumericalAspects}

Following the process of \cref{fig:PIC_140418_Thesis_Overview}, the models are tested in parametric studies.
The quality of the model is analyzed and the sensitivity on the crucial parameters is identified.
The numerical aspects are addressed in this chapter.\par
%
The influence of the number of \els in the simulations is analyzed in \cref{sec:SimulationElements}.
A minimal number for the simulations is found.
It sets the bounds on the accuracy of the model on the global values.\\
The difference between a smooth and a non-smooth simulation model is compared next.
The stiffness between \els and \rings is changed as well as their friction description.
Besides, the friction between \el-flanks and sheaves is analyzed, which yields conclusions on the non-smooth modeling for the system in \cref{sec:NonsmoothSimulations}.\\
In \cref{sec:RingModels} different \ring models are compared \wrt local and global effects.
It is shown, that a specific modeling for the \CVT dynamics improves the results.
A glance at the spatial outputs shows possibilities for future projects. 
A compromise between accuracy and computational cost has to be chosen for efficient simulation.\par

To draw general conclusions about the parametric influence, the boundary conditions are varied as well.
The settings are taken from measurements to ensure realistic settings.
Three different loads are applied at three different ratios.
In experiments the transmitted primary torque $M_P^1 = 30 \unit{Nm}$, $M_P^2 = 50 \unit{Nm}$ and $M_P^3 = 100 \unit{Nm}$ is varied.
The values for $M_S^*$ and $F_S^*$ result from measurements of a reference belt.
The measurements were conducted by \Bosch.
All simulations run with the same $\omega_P$, i.e. $2000\ \unit{\RPM}$, as the belt speed has a minor influence on the behavior.
Altogether, the nine \BCs are given in \cref{tab:BCCasesForParVar}.\\
The analysis follows the idea to present the global influence on the system behavior at first.
Mainly the thrust ratio \iF and the local efficiency \eff are given as the most important global measures.
A local analysis follows explaining the main findings.
Local and global output curves and values are utilized as examples.

\begin{table}[ht]
\centering
\begin{tabular}{l|c c c c c c c c c}
  \toprule
   & BC01 & BC02 & BC03 & BC04 & BC05 & BC06 & BC07 & BC08 & BC09\\
  \midrule
  $i_s [\unit{-}]$  & 0.47 & 1.00 & 2.42 & 0.47 & 1.00 & 2.42 & 0.47 & 1.00 & 1.30\\
  $M_P \unit{[Nm]}$ & 30   & 30   & 30   & 50   & 50   & 50   & 100  & 100  & 100\\
  \bottomrule
\end{tabular}
\caption[Boundary conditions used for the parameter variations]{Boundary conditions used for the parameter variations. All cases use $\omega_P = 2000$ \RPM and a safety value of $S \approx 1.3$.}
\label{tab:BCCasesForParVar}
\end{table}

%\input{NA_SimulationElements}
%\input{NA_NonSmoothSimulations}
%\input{NA_RingModels}