\chapter{Eulerian Description for a One Dimensional Closed Continuum} \label{chap:CoordinateTransformationForTheOneDimConti}

In the following the transformation between the Eulerian description and the Lagrange description is shown for a one dimensional continuum.
For the general case \textsc{Irschik} and \textsc{Holl} derive the equations \cite{irschik_equations_2002}.
The following derivation holds for a one dimensional closed continuum and thus is simpler to grasp.
\footnote{The derivation was done together with my co-worker Thorsten Schindler and I want to acknowledge his support and contributions \cite{grundl_ale_2015}.}

\section{Preliminaries}

The position of a particle along a curve $\barx$ is described with its original position $\xi$ and its motion along the curve $s(t)$ with
%
\begin{align}
  \barx = \xi + s(t), \qquad \frac{d \barx}{d \xi} = \frac{\partial \barx}{\partial s} = 1, \qquad \frac{d \barx}{dt} = \frac{d s}{dt} 
\end{align}
%
where $\xi$ is the Eulerian coordinate describing a fixed point in space and $\barx$ is the Lagrangian coordinate following the particle over time.\\
%
A function $f$ is defined as an integral over $g(\barx)$ with time depending limits $l = l(t), u = u(t)$.
%
\begin{align*}
  f \defined \int _{l} ^{u} g(\barx, t) d\barx.
\end{align*}
%
The absolute time derivative of $f$ is given by the Leibniz integral rule. %(Kaplan 1992, p.275) or \url{http://mathworld.wolfram.com/Integral.html}.
%
\begin{align}
  \ddt{f} = \pd{f}{l}  \ddt{l} + \pd{f}{u} \ddt{u} + \int _{l} ^{u}  \ddt{g} d \barx = -g(l) \ddt{l} + g(u) \ddt{u} +  \int _{l} ^{u} \pd{g}{\barx} \ddt{\barx} d \barx + \int _{l} ^{u} \pdt{g} d \barx
\end{align}

\section{Direct Approach}

Starting with the Lagrangian $\calL = \calT - \calV$ and the standard Lagrange equation
%
\begin{align*}
  \ddt{} \pd{\calL}{\dotqk} - \pd{\calL}{q_k} = 0
\end{align*}
%
a direct method to get the \eom can be derived.
With the kinetic energy 
%
\begin{align}
  \label{eq:General_Kinetic_Energy}
  \calT = \frac{1}{2} \int_\Omega v^T v \rho d \Omega = \frac{1}{2} \rho \int_\Omega v^Tv d\Omega
\end{align}
%
where $\Omega$ is the original volume for the integration, $\rho$ is the constant density of the material and $v$ the current velocity field, it follows the expression for the Lagrange equation.
%
\begin{align}
  \label{eq:Lagrange_Equation}
  \ddt{} \pd{\calT}{\dotqk} - \pd{\calT}{q_k} = Q _{k} \quad \text{with} \quad Q_k \defined - \frac{\partial \calV}{\partial q_k} 
\end{align}
%
For the given situation of a beam, the integral is constant in the two cross-section directions resulting in the area $A$.
Only the integral over the length from limit $l$ to limit $u$ has to be calculated along the beam in direction $\barx$.
%
%\begin{align}
%  T = \frac{1}{2} \rho A \int _{l} ^{u} v^Tv d\barx
%\end{align} 
%
Inserting this into the Lagrange equation, it follows
%
\begin{align}
  \ddt{} \pd{\calT}{\dotqk} &= \ddt{} \pd{}{\dotqk} \left( \frac{1}{2} \rho A \int _{l} ^{u} v^Tv d\barx \right) 
  = \frac{1}{2} \rho A  \left( \ddt{} \int _{l} ^{u} \pd{(v^Tv)}{\dotqk} d\barx \right) 
  =: \frac{\rho A}{2}  \ddt{} \int _{l} ^{u} g_v d\barx \nonumber  \\ 
  \label{eq:DerivativeOfKineticEnergyWithVariableBoundaries} 
  &= \frac{1}{2} \rho A \left( -g_v(l) \ddt{l} + g_v(u) \ddt{u} +  \int _{l} ^{u} \pd{g_v}{\barx} \ddt{\barx} d \barx + \int _{l} ^{u} \pdt{g_v} d \barx \right)
\end{align}
%
Remark that the limits of the integral do not depend on the generalized velocities $\dotqk$ but just on the transformation with $s$.\\

%\begin{remark}
%The function $g_v = g_v(\barx(s), q, \dotq)$ is a function of the Lagrange position, which depends on the drift $s$ as well might be a function of all other \dofs ($q, \dotq$).
%Thus it might depend on $s$ indirectly and maybe directly as $s$ is part of the \dofs.
%\end{remark}


\section{Closed Curve}
For a closed curve it holds $\ddt{l} = \ddt{u}$ and $g_v(l) = g_v(u)$.
Thus, the first two terms in \eqref{eq:DerivativeOfKineticEnergyWithVariableBoundaries} cancel each other.
Following the fundamental theorem of calculus, the first integral of it is zero as well.
%
\begin{align*}
  \int _l ^u \dot{s} \frac{\partial g_v}{\partial \barx} d \barx = \dot{s} (g_v(u) - g_v(l)) = 0
\end{align*}
%
A transformation of variables to integrate along $\xi$ with $u = L + s$ and $l = 0 + s$ and the length $L$ of the beam yields finally
%
\begin{align}
  \ddt{} \pd{\calT}{\dotqk} &= 
  %\frac{1}{2} \rho A  \int _{0} ^{L} \underbrace{\pd{g_v}{\xi}} _{= \pd{g_v}{\barx}} \dot{s} + \pdt{g_v} d \xi\\
  %&=
  %\underbrace{ \frac{1}{2} \rho A \dot{s}  \int _{0} ^{L}\pd{g_v}{\xi} d \xi } _{\text{First Term}} + 
  %\underbrace{ 
  \frac{1}{2} \rho A  \int _{0} ^{L}\pdt{g_v} d \xi
  % } _{\text{Second Term}}
  \label{eq:KineticEnergyTerms}
\end{align}
%
where the relative\footnote{The absolute time derivative is incorporated already. Only a direct time dependency is left and the relative time derivative is needed.} time derivative is needed.\\
Using the same ideas, the derivative of the kinetic energy \wrt. the generalized positions follows.
%
\begin{align*}
  \pd{\calT}{q_k} &= \frac{1}{2} \rho A \pd{}{q_k} \int _{l} ^{u} v^Tv d \barx = 
    \frac{\rho A}{2} \int _{0} ^{L} \pdqk{v^Tv} d \xi
\end{align*}
%
The potential energy can be represented as an integral over the volume $V \defined \int _{l} ^{u} g_V d\barx$ as well.
Thus, the last term of \eqref{eq:Lagrange_Equation} is defined as well.
%
\begin{align*}
  \pdqk{\calV} = \left(-g_V(l)  \pdqk{l} +  g_V(u) \pdqk{u} + \int _{l} ^{u} \pdqk{g_V} d \barx \right) = \int _{0} ^{L} \pdqk{g_V} d \xi
\end{align*}
%
Therefore, all expression are written in Eulerian description and for this special case of a closed beam structure the transformation does not yield additional terms. 

