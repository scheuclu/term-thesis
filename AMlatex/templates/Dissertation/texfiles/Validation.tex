\chapter{Validation} \label{chap:Validation}
Every model is an approximation of the reality.
Observations are written in mathematical formulas covering the main effects.
Numerical calculations, i.e. the simulations, compute these formulas only up to a certain accuracy.
A post-processing step calculates interpretable quantities out of the simulation results.\\
The range of uncertainty is discussed in the previous chapters as not all parameters are known.
It is therefore important to assess the following results together with the discussion in \cref{chap:NumericalAspects,chap:PhysicalAspects}.
In this chapter results of the present model are compared to measurements.
It shows the range of validity of the model.\par

Following the ideas of post-processing, the two main aspects, i.e. local and global results of the model, are compared.
A qualitative overview about the local forces is discussed at first.
Two further force distributions for varying boundary conditions are compared quantitatively.
Besides, local kinematics, i.e. spiral running and \el-\sheave pitch, yield further quantitative matches.\\
Four different variator geometries are used to compare global values while varying the boundary conditions.
Quantitative matches are gained for well identified parameters whereas parameter uncertainties only allow for qualitative discussion.
Further properties of the system are discussed qualitatively.
Slip-curves allow insights to the overall behavior of the system.
The endplay tolerance shows the system's robustness.\\
Finally, dynamical, i.e. instationary, properties are investigated as well.
On the one hand the scratch phenomenon is approached as \NVH issue.
On the other hand shifting dynamics are studied.