\chapter{Physical Aspects} \label{chap:PhysicalAspects}

The models sensitivity on numerical parameters is treated in the previous chapter.
Physical parameters are treated separately in this thesis.
Their value cannot be measured exactly, e.g. contact stiffness or friction.
Thus, they are varied to show the models sensitivity on them.
This serves two major goals.
On the one hand it shows the uncertainty of the model and thus the updating possibilities of it.
On the other hand it offers the basis for further improvement strategies for future design.\\
This chapter discusses the quality of the complete model on basis of the variations of the most crucial parameters.
It is intended to show possible variations of the model's output when the parameters are not known.
Again the nine cases of \cref{tab:BCCasesForParVar} are used to draw general conclusions.
They are sorted \wrt the ratio in \cref{tab:BCCasesForParVarSortedForRatio}.\par
%
\begin{table}[h]
\centering
\begin{tabular}{l l l}
  \toprule
  \OD & \MED & \UD\\ 
  \midrule
  1,4,7 & 2,5,8 & 3,6,(9)\\
  \bottomrule
\end{tabular}
\caption[The basis cases sorted with respect to the speed ratio]{The basis cases (\cref{tab:BCCasesForParVar}) sorted with respect to the speed ratio. The applied torque increases in each column with the number.}
\label{tab:BCCasesForParVarSortedForRatio}
\end{table}
%
Due to production uncertainties, the assembly conditions of the belt is not precisely known.
Thus, the influence of the endplay, i.e. the number of elements on the belt is studied at first in \cref{sec:RealElements}.
The connection between \els and pulleys is analyzed in \cref{sec:ElementPulleyContact}.
Different contact positions as well as the \el's axial stiffness show an impact on the overall systems behavior.
In contrast to the \RE, where the contact geometry is defined due to the circular shape, the contact geometry at the \head is not defined precisely.
The radial and the longitudinal variation of the contact point yields high influences on the system behavior as demonstrated in \cref{sec:ElementElementContact}.
Finally, the friction coefficients are analyzed.
Due to production and assembly the contact parameters vary, which changes the influence functions.
This is treated finally in \cref{sec:Friction}.\par

%\input{PA_RealElements}
%\input{PA_ElementPulleyContact}
%\input{PA_ElementElementContact}
%\input{PA_Friction}
