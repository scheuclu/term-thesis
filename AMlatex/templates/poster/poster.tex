% Compilation process: 
%     pdflatex poster.tex
\documentclass{AMposter}
\usepackage{pgfplots}
\usepackage{ngerman}

\title{Modellordnungsreduktion\\ für geometrisch nichtlineare Systeme}
\subtitle{Ein nichtlinearer Projektionsansatz}
\author{Johannes Rutzmoser}

\begin{document}
\maketitle

\begin{columns}
\column{0.5}

\block{Motivation \& Anwendungsgebiete}{
	\parbox[c]{0.48\linewidth}{
		\begin{tikzpicture}[scale=3.8]
	\draw [fill=TUMhighlightblue, TUMhighlightblue] (1,1) rectangle (4.8,4.8);
	\draw (2.9, 4.8) node [anchor = north, align = center, TUMblue] {online};
	\draw [->, color=black, very thick] (1,1) -- (5,1) node [below, align = right, anchor = north east] {Anzahl der Simulationen / \\ Simulationszeit};
	\draw [->, color=black, very thick] (1,1) -- (1,5) node [above, align=right, anchor = south east, rotate = 90] {Rechenaufwand};
	\draw [TUMred, fill opacity = 0.5, rounded corners=3pt, fill = TUMred, ] (0.9,0.9) rectangle (1.1,3.1); 
	\draw (1.3, 2) node [TUMred, rotate = 90, align = center] {offline};
	\draw [color=TUMblue, very thick] (1,1) -- (4.8,4.8);
	\draw [color=TUMblue, dashed, very thick] (1,1) -- (1,3) -- (4.8,4);
\end{tikzpicture}

	}
	\parbox[c]{0.48\linewidth}{
		\begin{itemize}\setlength\itemsep{1em}
			\item Regelungstechnik und\\ Echtzeitsysteme
			\item Optimierung
			\item Simulation von großen Modellen\\ bzw. über lange Zeiträume
			\item Substrukturen und\\ gekoppelte Probleme
		\end{itemize}
	}
}

\block{Überblick über nichtlineare Reduktionsmethoden}{
	\begin{tikzpicture}[xscale=19, yscale=3.2]
	\draw (1, 5)   node {\textbf{lineare Mannigfaltigkeit}};
	\draw (1, 4)   node [align = center]  {POD};
	\draw (1, 2)   node [align = center]  {Angereicherte Basis \\ des linearen Systems\\[0.5ex]\color{TUMgrey} Modale Ableitungen\\[0.5ex]\color{TUMgrey} Companion-Moden \\ \color{TUMgrey} \dots};
	\draw (2, 5)   node [align=center] {\textbf{nichtlineare quadratische Mannigfaltigkeit}};
	\draw (2.3, 4) node [] {Nichtlinear Guyan};
	\draw (2.3, 3) node [] {\color{TUMgrey} ebene Strukturen};
	\draw (1.7, 4) node [] {Modale Ableitungen};
	\draw (1.55, 3) node [align = center] {\color{TUMgrey} mit\\[-0.6ex]\color{TUMgrey} Trägheit};
	\draw (1.85, 3) node [align = center] {\color{TUMgrey} ohne\\[-0.6ex]\color{TUMgrey} Trägheit};
	\draw [thick, color=TUMblue] (1.35, 5) -- (1.35, 1);
	\draw [very thick, dashed, color=TUMblue, rounded corners=12pt] (2,4.5) -- (2.6, 4.5) -- (2.6, 1.1) -- (1.7, 1.1) -- (1.7, 3.5) -- (2, 3.5) -- cycle;
	\draw (2.15, 1.8) node [align = center] {\color{TUMblue} statische \\ \color{TUMblue}Kraftkompensationsmethode};
\end{tikzpicture}
}

\block{Lineare projektive Modellordungsreduktion}{
	Das nichtlineare dynamisches System zweiter Ordnung
	\begin{gather*}
		\vM\ddot{\vx} + \vD\dot{\vx} + \vf(\vx) = \vF(t)
	\intertext{wird mit linearer Transformation}
		\vx=\vV\vz
	\intertext{zu dem System mit der reduzierten Variable $\vz$ mit Residuum $\vr$ überführt:}
		\vM\vV\ddot{\vz} + \vD\vV\dot{\vz} + \vf(\vV\vz) = \vF(t) + \vr
	\end{gather*}
	
	\bigskip
	Forderung, dass Residuum orthogonal zu Spaltenraum von $W$ ist:
	\begin{gather*}
		\vW^T\vr = 0 \\
		\underbrace{\vW^T\vM\vV}_{\vM^*}\ddot{\vz} + \underbrace{\vW^T\vD\vV}_{\vD^*}\dot{\vz} + \underbrace{\vW^T\vf(\vV\vz)}_{\vf^*(\vz)} = \underbrace{\vW^T\vF(t)}_{\vF^*(t)}
	\end{gather*}
}

\block{Nichtlineare projektive Modellordungsreduktion}{
	Nichtlineares mechanisches System (ohne Dämpfung)
	\begin{equation*}
		\vM\ddot{\vx} + \vf(\vx) = \vF(t)
	\end{equation*}

	nichtlineare Transformation $\vx = \vg(\vz)$
	
	entspricht Projektion auf zustandsabhängige Basis $\vV(\vz)$
	
	ergibt die Differentialgleichung für das nichtlinear projizierte System:
	\begin{equation*}
		\pdiff{\vg}{\vz}^T \vM \pdiff{\vg}{\vz}\ddot{\vz} + 
		{\color{TUMblue}
			\underbrace{
				\frac{\partial \vg}{\partial\vz}^T \vM \frac{\partial^2 \vg}{\partial\vz^2} \dot{\vz}\dot{\vz}
			}_{\text{zusätzlicher konvektiver Term}}
		}
		+ \pdiff{\vg}{\vz}^T \vf\bigl(\vg(\vz)\bigr) = \pdiff{\vg}{\vz}^T \vF(t)
	\end{equation*}
}

\column{0.5}

\block{Nichtlineare statische Kondensation}{
	Idee der statischen Kondensation (Guyan): Elimination von einseitig gekoppelten Freiheitsgraden bei Vernachlässigung der dynamischen Effekte:\\
	Die Gleichung 
	\begin{gather*}
		\vM\ddot{\vx} + \vK\vx + \vf(\vx) = \vF(t) \\[1ex]
		\pmat{\vM_{11} & 0 \\ 0 & \vM_{22}} \pmat{\ddot\vx_1 \\ \ddot\vx_2} + 
		\pmat{\vK_{11} & \vK_{12} \\ \vK_{21} & \vK_{22}} \pmat{\vx_1 \\ \vx_2} + \pmat{\vf_1(\vx_1, \vx_2) \\ \vf_2(\vx_1)} = \pmat{\vF_1(t) \\ \vF_2(t)} 
	\intertext{kann mit der Bedingung}
		\vM_{22} \ddot{\vx}_2 = 0
	\intertext{und der Elimination von $\vx_2$ durch}
		\vx_2 = -\vK_{22}^{-1} \bigl[ \vK_{21}\vx_1 + \vf_2(\vx_2) - \vF_2(t) \bigr]
	\intertext{überführt werden in}
		\vM_{11} \ddot{\vx}_1 + \vK_{11}\vx_1 - \vK_{12}\vK_{22}^{-1} 
		\bigl[ \vK_{21}\vx_1 + \vf_2(\vx_1) - \vF_2(t) \bigr]
	\end{gather*}

	\bigskip
	Idee der nichtlinearen statischen Kondensation: Projektion auf die Kondensationsmannigfaltigkeit. Beispiel einfacher Balken:\\
	Bewegungsgleichung:
		\begin{equation*}
			\underbrace{
				\pmat{\vM_{w} & 0 \\ 0 & \vM_{u}}
			}_{\vM} 
			\pmat{\ddot{\vw} \\ \ddot{\vu}} + 
			\underbrace{
				\pmat{\vK_{w} & 0 \\ 0 & \vK_{u}}
			}_{\vK} 
			\pmat{\vw\\\vu} + 
			\underbrace{
				\pmat{\vf_w(\vw, \vu) \\ \vf_u(\vw)}
			}_{\vf(\dots)} 
			= 
			\underbrace{
				\pmat{\vF_w(t) \\ \vF_u(t)}
			}_{\vF(t)}
		\end{equation*}
		Projektion auf Mannigfaltigkeit liefert den Projektor:
		\begin{equation*}
			\vK_u \vu + \vf_u(\vw) = 0 \qquad \Longrightarrow\quad
			\pmat{\vw \\ \vu} = \vg(\vw) = \pmat{\vw \\ -\vK_{u}^{-1} \vf_u(\vw)}
		\end{equation*}
		Reduzierte Bewegungsgleichung:
		\begin{equation*}
			\pdiff{\vg}{\vw}^T \vM \pdiff{\vg}{\vw} \ddot{\vw} + \pdiff{\vg}{\vw}^T \vM \pdiff[2]{\vg}{\vw} \dot{\vw}\dot{\vw} + \pdiff{\vg}{\vw}^T \vK\vg(\vw) + \pdiff{\vg}{\vw}^T \vf\bigl(\vg(\vw)\bigr) = \pdiff{\vg}{\vw}^T\vF(t)
		\end{equation*}
}


\block{Statische Kraftkompensationsmethode}{
	The linear Basis $\vV$ kann frei gewählt werden. Sie beschreibt die Abbildung an der Stelle $z=0$: 
	\begin{equation*}
		\vV = \pdiff{\vx}{\vz} \biggr|_{\vz=0}
	\end{equation*}
	Die Idee ist, die lineare Abbildung zu einer Abbildung zweiter Ordnung zu expandieren:
	\begin{equation*}
		\vg(\vz) = \pdiff{\vx}{\vz} \biggr|_{\vz=0} \vz + \frac{1}{2} \pdiff[2]{\vx}{\vz} \biggr|_{\vz=0} \vz\;\vz
	\end{equation*}
	
	Nun wird gefordert, dass sich die Abbildung so verhält, dass die zweite Ableitung der Kraft im transformierten System null ist:
	\begin{equation*}
		\pdiff[2]{}{\vz} \Bigl( \vK\vx + \vf_{nl}(\vx) \Bigr) \biggr|_{\vz=0} = 0
	\end{equation*}
	daraus folgt die zweite Ableitung der Abbildung zwischen $x$ und $z$:
	\begin{equation*}
		\pdiff[2]{\vx}{\vz} \biggr|_{\vz=0} = - \vK^{-1} \pdiff[2]{\vf_{nl}}{\vz} \biggr|_{\vz=0} = -\vK^{-1} \pdiff[2]{\vf_{nl}}{\vx} \biggr|_{\vx=0} \;\vV\;\vV
	\end{equation*}
	
	So ergibt sich der nichtlineare Projektor zu:
	\begin{align*}
		\vg(\vz) = & \vV \vz - \vK^{-1}\, \pdiff[2]{\vf_{nl}}{\vx} \Bigr|_{\vx=0} \;(\vV\vz)\;(\vV\vz) \\
		= & \vK^{-1}\left(\vK \vV \vz + \pdiff[2]{\vf_{nl}}{\vx} \Bigr|_{\vx=0} \;(\vV\vz)\;(\vV\vz)\right)
	\end{align*}
}

\end{columns}

\end{document}
